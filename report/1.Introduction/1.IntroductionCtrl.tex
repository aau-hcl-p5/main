% !TeX root = ../main.tex
% This chapter is aimed to introduce both the subject of the project aswell as our motivation for the project.
% It also aims to establish the intitial problem statement, which the analysis will be conducted based on.
% This section should describe the basis of our project.
% really where our solution originates, based on a real world problem,
% even though our project is based on a "fun" project, it should still be possible to abstract it into real world applications

\chapter{Introduction}

Tracking airborne objects can be quite difficult.
This difficulty has inspired different sports and hobbies such as skeet shooting.
Hitting a target requires tracking of it, its velocity and direction, as well as the calculation or prediction of its future trajectory.
The trajectory can be impacted by a number of different factors, including gravity, wind, initial force, or any artificial acceleration.
Furthermore, if the object is intelligent in any way, the path might change procedurally.
%TODO procedurally is a bad word for this, isn't it?

Tracking of airborne objects has multiple different applications.

\subsection{Airports}
An estimated thirteen thousand "bird strikes" happen each year, which is the event of a bird colliding with a vehicle, primarily aircrafts\cite{WikiBirdStrike}\cite{WildlifeStrikeReport}.
%TODO can we get rid of this wikipedia source? it makes me angry
Although 65\% are estimated to cause somewhat insignificant damage, the damages are estimated to cost 1.2 billion dollars each year, world-wide \cite{CostOfBirdstrikes}.
The majority of these accidents happen at low altitude, during landing or takeoff\cite{CostOfBirdstrikes}.
Airports utilize different methods of bird control, including killing, trapping, poison and lights\cite{BirdControlAtAirports}.
Shooting birds is one of the most commonly used methods, and has been observed to be quite effective, while also reinforcing other methods.
This both eliminates the individual bird, but also scares away other birds.
However, it can be expensive in both salaries and effort.

\subsection{Military}
In a military setting, hitting a moving airborne target is relevant in a defense context, for instance destroying incoming projectiles or shooting down hostile planes.
Compared to the airport-case, the problem can arise at a variety of locations, both on vehicles, boats, planes and tanks, but also on stationary military installations.
The American military developed the Phalanx CIWS, a close-in weapon system, which is a radar-guided 20mm Gatling-style rotary cannon, mounted on a base that moves on two axes\cite{PhalanxCIWS}.
These are mounted on ships and can shoot incoming missiles and other airborne objects deemed a threat.
The Dutch also developed a similar system called the Goalkeeper CIWS.
%TODO The Dutch...... military?
Both of these solutions are autonomous.

\subsection{Initial Problem}

Tracking an airborne object has multiple different problem areas that makes it an interesting domain to research.
This report will try to look into ways of solving this problem on a smaller scale, in clinical examples, and to research the different parameters.
Before hitting an airborne target, it is important to estimate its trajectory, and the problem can therefore be split into two sub-problems.
\begin{enumerate}
  \item Calculating the time it takes before a projectile will reach the target.
  \item Calculating where the target will be in T (mili)seconds.
\end{enumerate}
Combined, these two sub-problems means finding the spot where the travel time of the target meets the travel time of the projectile.

This report is build on courses in both real-time systems as well as machine learning, and this problem works in the context of both.
%TODO is build on sounds a bit silly
In regards to machine intelligence, the problem of tracking an airborne target has potential for intelligent analysis by an artificial agent.
Likewise, it is a time-critical task to move a weapon so that it aims at the target and hits the target, before the target is out of range or hits the ground.

As hitting the target with a projectile introduces a series of unknown variables, this report will focus on tracking the target with a laser, as the travel time is near instant, and it's less affected by gravity.
The report will use more simplistic targets, to remove unpredictable parameters, and to remove the possibility of harming any living creatures.

These considerations creates the following initial problem, whereas the analysis will lead to a more in-depth set of requirements and a concrete problem statement.
\begin{center}
  \textit{\textbf{How can a software system track a moving airborne projectile?}}
\end{center}
