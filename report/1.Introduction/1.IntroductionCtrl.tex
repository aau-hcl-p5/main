% !TeX root = ../main.tex
% This chapter is aimed to introduce both the subject of the project as well as our motivation for the project.
% It also aims to establish the initial problem statement, which the analysis will be conducted based on.
% This section should describe the basis of our project.
% really where our solution originates, based on a real world problem,
% even though our project is based on a "fun" project, it should still be possible to abstract it into real world applications

\chapter{Introduction}
Tracking airborne objects can be quite difficult, as hitting an airborne target requires tracking the object, along with its velocity and direction, as well as the calculation or prediction of its future trajectory.
The trajectory can be impacted by a number of different factors, including gravity, wind, initial force, or any artificial acceleration.
Furthermore, if the object is intelligent it might change its path, rendering the prediction of the trajectory moot.
Tracking of airborne objects has multiple different applications.

\subsection{Airports}
An estimated thirteen thousand \textit{bird strikes} happen each year, which is the event of a bird colliding with a vehicle, primarily aircrafts\cite{WildlifeStrikeReport}.
Although 65\% are estimated to cause insignificant damage, the damages are estimated to cost 1.2 billion dollars each year, world-wide \cite{CostOfBirdstrikes}.
The majority of these accidents happen at a low altitude, meaning during landing or takeoff\cite{CostOfBirdstrikes}.
Airports utilize different methods of bird control, including killing, trapping, poison and lights\cite{BirdControlAtAirports}.
Shooting birds is one of the most commonly utilized methods, and has been observed to be effective in combination with reinforcing other methods.
This both eliminates the individual bird, but also scares away other birds, which serves as a form of bird strike prevention.
However, the downside are the expenses in both salaries and effort.

\subsection{Military}
In a military setting, hitting a moving airborne target is relevant in a defense context, such as destroying incoming projectiles or shooting down hostile planes.
Compared to the airport case, the problem can arise at a variety of locations, both on vehicles, boats, planes and tanks, but also on stationary military installations.
The American military developed the Phalanx CIWS, a close-in weapon system, which is a radar-guided 20mm Gatling-style rotary cannon, mounted on a base that moves on two axes.
These are mounted on ships and can shoot incoming missiles and other airborne objects deemed a threat.
Alternatives to the Phalanx exist, for example the Goalkeeper, so this is not a unique concept, and most of these rely on similar principles.
Both the Goalkeeper and Phalanx are autonomous systems.

\subsection{Initial Problem}
Tracking an airborne object has multiple different problem areas, making it an interesting domain to research.
This report will look into ways of solving the aforementioned problem on a smaller scale, in clinical examples, and to research the different parameters hereof.
Before hitting an airborne target, it is important to estimate its trajectory, and the problem can therefore be split into two sub-problems:
\begin{enumerate}
  \item Calculating the time it takes before a projectile will reach the target.
  \item Calculating where the target will be in T milliseconds.
\end{enumerate}
Combined, these two sub-problems means finding the spot where the travel time of the target meets the travel time of the projectile.

As the goal of this semester project is to apply the knowledge from the real-time systems and machine intelligence courses, the problem that the project seeks to solve, should be within the domain of both, which is the case with the trajectory prediction problem.

In regards to machine intelligence, the problem of tracking an airborne target has potential for intelligent analysis by an artificial agent.
Likewise, it is a time-critical task to move a weapon so that it aims at the target and hits the target, before the target is out of range or hits the ground.

The report will use simplified targets to remove unpredictable parameters, and to remove the possibility of harming any living creatures.

These considerations leads to the following initial problem, whereas the analysis will lead to a more in-depth set of requirements and a concrete problem statement:
\label{key:initialProblem}

\begin{center}
  \textit{\textbf{How can a software system track a moving airborne target?}}
\end{center}
