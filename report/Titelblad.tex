\newpage
\makeatother
\begin{minipage}[T]{0.45\textwidth}
 \begin{flushleft}
  \textbf{\normalsize{Title:}}\\ \maketitle 
  \textbf{\normalsize{Semester:}}\\5\\
  \textbf{\normalsize{Project theme:}}\\Embedded Systems\\
  \textbf{\normalsize{Project duration:}}\\04/09/2018 - 10/12/2018\\
  \textbf{\normalsize{ECTS:}}\\15\\
  \textbf{\normalsize{Supervisor:}}\\??\\

  \large{\textsf{\textbf{\normalsize{Authors:}}}}\\
  [1ex]
  \begin{tabular}{ll}
   \normalsize{Casper Weiss Bang}\\
   \makebox[2.4in]{\hrulefill}\\
   \normalsize{Daniel Moesgaard Andersen}\\
   \makebox[2.4in]{\hrulefill}\\
   \normalsize{Nichlas Ørts Lisby}\\
   \makebox[2.4in]{\hrulefill}\\
   \normalsize{Thomas Lundsgaard Hansen}\\
   \makebox[2.4in]{\hrulefill}\\
   \normalsize{Thomas Højriis Knudsen}\\
   \makebox[2.4in]{\hrulefill}\\
   \normalsize{Frederik Spang Thomsen}\\
   \makebox[2.4in]{\hrulefill}\\
  \end{tabular} 
 \end{flushleft}
\end{minipage}
 ~
\begin{minipage}[T]{0.45\textwidth}
 \begin{flushright}
  \includegraphics[width=0.9\textwidth]{images/aau_logo.pdf}\\[1.0 cm]
 \end{flushright}
 \begin{flushleft}
  \textbf{Abstract:}\\
  In the Danish educational system, the Arduino platform is often used to introduce students to computer science. 
  The Arduino's hardware is relatively simple and easy to grasp. 
  However, the Arduino language can often be difficult for novices.
  While the Arduino language is a simplified subset of \texttt{C++}, there are still parts of the language that are difficult to comprehend for novices.
  
  This report elaborates and explains the development of the HCL language, a programming language designed to ease the introduction to programming for students and novices, whilst also implementing high-order functionality on the Arduino.
  HCL does this by emphasizing a resemblance to the English language.
  A compiler for HCL was developed, as part of the project.
  The compiler is written in Kotlin, by hand, and compiles to \texttt{C++} code.
  
  %\scriptsize{}
  
 \end{flushleft}
\end{minipage}\\
Number of pages: \ref{TotPages}.\\
\begin{center}
 \begin{scriptsize}
  \textbf{\underline{Everyone is permitted to copy and distribute verbatim copies of this document,}}\\ \textbf{\underline{ meaning that changing it is prohibited.}}
 \end{scriptsize}
\end{center}
