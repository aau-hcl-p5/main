\section{Hardware testing}\label{des:sec:hwtest}
To ensure the best possible accuracy of the device, it is important to ensure that the quality and precision of the chosen hardware is at its highest.
Therefore, a test of the motors and the chosen camera has been conducted, checking their precision and quality.

\subsection*{Motors}
To ensure the capabilities of the motors, the tests have to try out combinations of target angles and movement speed, and to what extent precision is reached, considering the braking distance.
Therefore a setup combining an angle, and a velocity ranging from 0 to 100 is setup as seen in Appendix~\ref{appendix:motortest}.

The terms \textit{stable}, \textit{small correction} and \textit{huge correction} are used to specify the stopping at the angle, and to elaborate on, not only precision, but the amount of correction needed to stop at the exact angle specified.

The goal of the tests is to start at a zero degree angle, speed up to the specified velocity, and stop when reaching the target angle; just in case the motor will go further, it will automatically stop and set the speed in the opposite direction, until reaching its goal, hence the correction.

As the motors are used to aim, precision is a highly valued quality.
This means that the motors used, and the movement speed should be able to stop at a certain angle, when running at speed x.
An alternative is to have a built-in correction, that will give the motors a varying velocity.

The expected result and the actual result of the test are not as surprising: The higher the speed, the more correction is required.
However, it does not appear to be affected by the angle to which it has to aim, which is also observed in the testing; acceleration to the target speed happens rather fast.

In conclusion, in case we want to achieve the highest possible aim-speed, we should take action upon the correction at higher speeds and move at lower speeds when the motor approaches the target to reduce the braking distance, thus avoiding corrections.
The results of the tests are therefore used to not only choose which motors to use, but also gives a hint towards what the optimal movement speed is.

\subsection*{Camera}
Due to the use of the camera, the actual quality of the footage is secondary.

The importance criteria of the camera is that it is able to keep the blur to a minimum, with a moving target on the video feed.
This renders the resolution irrelevant.

Furthermore, is the framerate irrelevant to a certain degree, depending on the speed of our software, and its capabilities, as long as the camera provides a higher framerate, than the software processes.

To accomodate this, the camera test has been consistent of a manual recording of different USB Web Cameras, and looking at moving targets, frame by frame, to find the one with the least blurry images between movements.
