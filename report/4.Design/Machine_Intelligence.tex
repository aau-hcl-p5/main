\section{Modelling the agent}\label{Design:MI}
Before developing an intelligent device, it is important to analyze the different aspects of the agent and the world, based on the principles described in Section~\todo{link to mi part of theory}.


In the context of this project, the agent is the \texttt{F.L.A.T}.
Its goal is to point at a given target and fire a laser beam, whereas the \textit{ability} of agent, is to rotate its motors.
Its only \textit{stimuli} is a webcam, as concluded in Section~\ref{sec:obj_tracking}.

The environment is the surroundings of the turret, where the \texttt{F.L.A.T} can be considered the sole purposeful agent in the world, whereas the target can be seen as nature.
In a more advanced case, the targets can also be considered purposeful agents.
An example could be how birds might flee, or incoming projectiles could be designed to elude the turret.

The optimal solution consists of a set of actions that lowers the distance from the target to the center of the webcam.
This introduces three potential problems, based on the analysis and the hardware tests.



\subsection{Object localization}
Object localization is determining where the target is relative to the agent.
The most obvious option, and the option that we expected to apply from the beginning, is object localization, which were analyzed in Section~\ref{sec:anal:objdet}.
It is a problem that is required to be solved. The agent cannot move in the direction of the target, if the target location is not known.
This issue would ideally be solved by online learning, as to make sure that the device continuously is able to find the target even in an environment where the visible aspects change, as lighting change and as the physical device might move. 
This might be expensive regarding computational power especially as a visual input can include a large set of data.
These considerations will be explored in depth in Section~\ref{Design:ObjectLocalization}.

\subsection{Target trajectory prediction}
Meaning the predicting where the target is going to be.
In Subsection~\ref{sec:obj_tracking:sub:ML}, it was mentioned that a potential application of machine intelligence was in trajectory prediction for the located object.
While a thrown object would be quite simple to predict, objects like birds or planes might move according to different patterns.
An MI model would then be able to predict the movement of the object.
This would allow the \texttt{F.L.A.T} to simply shoot at the predicted location, instead of continuously tracking the object.
This approach would also be continuously refined, if targets hit was tracked as well.
A disadvantage to this approach is that object classification would also be required, to allow distinction of the movement of different objects.
This would allow for an iterative approach, assuming object classification and localization was implemented first.

\subsection{Variable motor power}
Variable motor power is finding out how much power is needed when moving on the $Y$-axis.
An interesting result from the motor tests done in Section~\ref{des:sec:hwtest}, was that motors with more weight, required more power to move the same distance as a motor without added weight.
This also includes the additional force of gravity, as a motor moving downwards, required less power than a motor that moved upwards.
Additionally, the wires that connect the motors to the NXT added additional variable weight, depending on how loose or tense the wire was.
This is a problem, as it could potentially have an impact on precision, which is important when the goal is to hit a fast moving object.

A potential solution would be to apply machine learning to this aspect of the system.
Our initial thought is to add an additional calibration step upon startup of the \texttt{F.L.A.T}.
This calibration step would be used to extract data from the machine.
Based on the data, a neural network would then be trained, allowing the \texttt{F.L.A.T} to apply variable power, depending on the angle of the motor.
The result from the model would be a prediction of the amount of power required to move the motor the given desired angle.


\subsection{Conclusion}\label{Design:MI:sub:conclusion}
Each problem requires its own independent solution.
%Definier i mit hvad satisfactory solution og optimal solution er.
%Tror det skal være noget i stil med:
%For this project the ideal solution is a \textit{satisfactory solution}, not necessarily the \textit{optimal solution}, due to the limitations of the hardware and time constraints.
%den sidste smule er ligegyldig nok
The ideal solutions is a \textit{satisfactory solution}, not necessarily the \textit{optimal solution}, due to the limitations of the hardware and the development time.
%tilføj offline, online and design time til min section
To build an intelligent agent, it important to consider whether the reasoning happens at design time, offline, or online.
%Noget i stil med:
%Hvad er critical tho, tror det skal uddybes.
%pas på med nedenstående foreslag, tror det er wack det jeg har skrevet.
%This is dependent on a couple of factors, one being the computation time of reasoning in real-time, which can be critical, and because of limited hardware, it is ideal to compute as many calculations early, to ensure that deadlines are met.
%undgå our, we osv.
This depends on varying factors, for instance the computation time of reasoning in real-time, which can be critical, and due to the limited hardware, it is ideal to do as many calculations as early as possible, to make sure we reach our deadlines.
%Tror det skal være MI problems, da det var det vi skrev ovenover.
The three issues will be considered with the above in mind.

All three of these options are great, as they would be viable applications of machine learning.
However, as the target trajectory prediction method also requires object classification, we will not be looking further into it, at this moment.
The method that we are most keen on working with is object localization.
In the following section, we will be looking into different methods of applying object localization, as well as testing the performance on our hardware platform.
