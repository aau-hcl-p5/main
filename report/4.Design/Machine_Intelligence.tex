\section{Machine Intelligence}\label{Design:MI}
Machine intelligence is a formal requirement for the project.
This section will explore the different options for machine intelligence.

We currently see 3 different possibilities for machine intelligence.

\subsection{Object localization}
The most obvious option, and the option that we expected to apply from the beginning, is object localization.
Object localization has already been discussed in Section~\ref{sec:anal:objdet}.
Simply put, it would be to train an agent to recognize different objects, with the goal of returning a location.
As this has already been explored quite extensively, it will not be elaborated further upon.

\subsection{Target trajectory prediction}
In Subsection~\ref{sec:obj_tracking:sub:ML}, it was mentioned that a potential application of machine intelligence was in trajectory prediction for the located object.
While a thrown object would be quite simple to predict, objects like birds or planes might move according to different patterns.
An MI model would then be able to predict the movement of the object.
This would allow the \texttt{F.L.A.T} to simply shoot at the predicted location, instead of continuously tracking the object.
This approach would also be continuously refined, if targets hit was tracked as well.
A disadvantage to this approach is that object classification would also be required, to allow distinction of the movement of different objects.
This would allow for an iterative approach, assuming object classification and localization was implemented first.

\subsection{Variable motor power}
An interesting result from the motor tests done in Section~\ref{des:sec:hwtest}, was that motors with more weight, required more power to move the same distance as a motor without added weight.
This also includes the additional force of gravity, as a motor moving downwards, required less power than a motor that moved upwards.
Additionally, the wires that connect the motors to the NXT added additional variable weight, depending on how loose or tense the wire was.
This is a problem, as it could potentially have an impact on precision, which is important when the goal is to hit a fast moving object.

A potential solution would be to apply machine learning to this aspect of the system.
Our initial thought is to add an additional calibration step upon startup of the \texttt{F.L.A.T}.
This calibration step would be used to extract data from the machine.
Based on the data, a neural network would then be trained, allowing the \texttt{F.L.A.T} to apply variable power, depending on the angle of the motor.
The result from the model would be a prediction of the amount of power required to move the motor the given desired angle.

\subsection{Conclusion}\label{Design:MI:sub:conclusion}
All three of these options are great, as they would be viable applications of machine learning.
However, as the target trajectory prediction method also requires object classification, we will not be looking further into it, at this moment.
The method that we are most keen on working with is object localization.
In the following section, we will be looking into different methods of applying object localization, as well as testing the performance on our hardware platform.
