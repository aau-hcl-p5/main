\section{Machine Intelligence}\label{Design:MI}

\subsection{Object localization}
An early working product was favored in order to get a baseline for all components to be further developed from.
Therefore since object localization using machine learning algorithms requires a bit of setup, along with a lot of training data some algorithms were implemented without machine learning.

\paragraph{Zone Average}
Initially an object detection algorithm dubbed "Zone Average" were created.
Zone Average locates the most red object in an image and.
It works by splitting the image into $N\times N$ tiles, and determining the total redness within each tile.
The tile that contains the most red is determined to contain the object, and the location of that tile is returned as a result.

In order to increase accuracy, dynamic resizing of tiles were introduced.
The idea is that if the same tile has been the selected tile for multiple iterations, the size of the tiles should decrease to get better accuracy.
Likewise when the object begins moving the size of the tiles should expand.

\figur{0.6}{images/zone_avg.png}{Zone average algorithm detecting red balloon.}{fig:zone_avg}

Zone Average functioned well as a starting point, but it lacked accuracy as also illustrated in image \ref{fig:zone_avg} where multiple tiles could be very red, but not necessarily in the center.

\paragraph{Object Fill}
Following Zone Average a new algorithm called "Object Fill" was created.
Object Fills purpose was to render very accurate position results.


\figur{0.6}{images/obj_fill.png}{Object Fill algorithm detecting red balloon.}{fig:obj_fill}

\paragraph{Thresh Moment}

\figur{0.8}{images/thresh_moment.png}{Thresh Moment algorithm detecting red balloon.}{fig:thresh_moment}

\paragraph{YOLO}

\figur{0.6}{images/yolo_example.png}{YOLO algorithm detecting person.}{fig:yolo_example}
