\section{Modelling the agent}\label{Design:MI}
Before developing an intelligent device, it is important to analyse the different aspects of the agent, the \texttt{F.L.A.T system} and the domain, based on the principles described in Section~\todo{link to mi part of theory}.


The agent has a singular \textbf{goal} of pointing a laser at a ball or balloon.
The \textbf{ability} to rotate on horizontal axis and vertical axis.
Its only \textbf{stimuli} is a video camera, as concluded in Section~\ref{sec:obj_tracking}.
Therefore, the \textbf{environment} is a physical one, but it can be considered to be the only purposeful agent in the world.
In a more advanced case, the target can be considered a purposeful agent - you could imagine that birds would flee, or incoming rockets would be designed to elude the turret, however as our target is either a balloon or a ball, the actions of the target are quite predictable.

The purpose of this is to develop an intelligent agent, which acts according to it's environment, and with it reaches it's goals.
In more concrete terms, through its abilities, the robot should lower the distance between the crosshair, the centre of the webcam, and the target.
This introduces three different problems from an MI perspective, based on the analysis and the hardware tests.

\subsection{Object localization}
The most obvious option, and the option that we expected to apply from the beginning, is object localization, which were analyzed in Section~\ref{sec:anal:objdet}.
It is a problem that is required to be solved. The agent cannot move in the direction of the target, if the target location is not known.
This issue would ideally be solved by online learning, as to make sure that the device continuously is able to find the target even in an environment where the visible aspects change, as lighting change and as the physical device might move. 
This might be expensive regarding computational power especially as a visual input can include a large set of data.
These considerations will be explored in depth in Section~\ref{Design:ObjectLocalization}.

\subsection{Target trajectory prediction}
In Subsection~\ref{sec:obj_tracking:sub:ML}, it was mentioned that a potential application of machine intelligence was in trajectory prediction for the located object.
While a thrown object would be quite simple to predict, objects like birds or planes might move according to different patterns.
An MI model would then be able to predict the movement of the object.
This would allow the \texttt{F.L.A.T} to simply shoot at the predicted location, instead of continuously tracking the object.
This approach would also be continuously refined, if targets hit was tracked as well.
A disadvantage to this approach is that object classification would also be required, to allow distinction of the movement of different objects.
This would allow for an iterative approach, assuming object classification and localization was implemented first.

\subsection{Variable motor power}
An interesting result from the motor tests done in Section~\ref{des:sec:hwtest}, was that motors with more weight, required more power to move the same distance as a motor without added weight.
This also includes the additional force of gravity, as a motor moving downwards, required less power than a motor that moved upwards.
Additionally, the wires that connect the motors to the NXT added additional variable weight, depending on how loose or tense the wire was.
This is a problem, as it could potentially have an impact on precision, which is important when the goal is to hit a fast moving object.

A potential solution would be to apply machine learning to this aspect of the system.
Our initial thought is to add an additional calibration step upon startup of the \texttt{F.L.A.T}.
This calibration step would be used to extract data from the machine.
Based on the data, a neural network would then be trained, allowing the \texttt{F.L.A.T} to apply variable power, depending on the angle of the motor.
The result from the model would be a prediction of the amount of power required to move the motor the given desired angle.


\subsection{Conclusion}\label{Design:MI:sub:conclusion}
Each problem requires its own independent solution.
The ideal solutions is a \textit{satisfactory solution}, not necessarily the \textit{optimal solution}, due to the limitations of the hardware and the development time.
To build an intelligent agent, it important to consider whether the reasoning happens at design time, offline, or online.
This depends on varying factors, for instance the computation time of reasoning in real-time, which can be critical, and due to the limited hardware, it is ideal to do as many calculations as early as possible, to make sure we reach our deadlines.
The three issues will be considered with the above in mind.

All three of these options are great, as they would be viable applications of machine learning.
However, as the target trajectory prediction method also requires object classification, we will not be looking further into it, at this moment.
The method that we are most keen on working with is object localization.
In the following section, we will be looking into different methods of applying object localization, as well as testing the performance on our hardware platform.
