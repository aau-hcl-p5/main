\section{Modelling the agent}\label{Design:MI}
Before developing an intelligent device, it is important to analyze the different aspects of the agent and the world, based on the principles described in Section~\todo{link to mi part of theory}.


In the context of this project, the agent is the \texttt{F.L.A.T}.
Its goal is to point at a given target and fire a laser beam, whereas the \textit{ability} of agent, is to rotate its motors.
Its only \textit{stimuli} is a webcam, as concluded in Section~\ref{sec:obj_tracking}.

The environment is the surroundings of the turret, where the \texttt{F.L.A.T} can be considered the sole purposeful agent in the world, whereas the target can be seen as nature.
In a more advanced case, the targets can also be considered purposeful agents.
An example could be how birds might flee, or incoming projectiles could be designed to elude the turret.

The optimal solution consists of a set of actions that lowers the distance from the target to the center of the webcam.
This introduces three potential problems, based on the analysis and the hardware tests.
These problems will first be analyzed in subsections
The following subsections  will explore the problems, and evaluate whether these problems has a relevance in the context of developing the \texttt{F.L.A.T} system. 


\subsection{Object localization}
Determining where the target is relative to the agent.


The most obvious option, and the option that we expected to apply from the beginning, is object localization, which were analyzed in Section~\ref{sec:anal:objdet}.
It is a problem that is required to be solved, the agent cannot move in the direction of the target, if the target location is not known.
This issue would ideally be solved by online learning, as to make sure that the device continuously is able to find the target even in an environment where the visible aspects change, as lighting change and as the physical device might move. 
This might be expensive regarding computational power especially as a visual input can include a large set of data.
These considerations will be explored in depth in Section~\ref{Design:ObjectLocalization}.

\subsection{Target trajectory prediction}
The predicting of the future location of the target.


In Subsection~\ref{sec:obj_tracking:sub:ML}, trajectory prediction was mention as a  a potential application of machine intelligence.
This essentially means predicting the actions of the target, if it is considered to be an agent, makes it relevant whether the target is purposeful.
As the target is a ballon it is an inanimate object, and should not be considered as an intelligent agent.
That means the motion will follow the laws of physics, as the trajectory can be simplified to a velocity and a constant force vector based on gravity.
If the target was a piece of paper, air resistance might be an issue, but generally speaking balls follows a simpler path.

Even though this simplifies prediction, the relevance is still worth considering.
Prediction is relevant if shooting is a non-instant process, as the travel time of a projectile has to be considered.
This has previously been defined as being out of scope, so this is not worth noting.
Finally it might be relevant to move to the future location of the target, to catch up with a fast moving target.
However, this did not seem to have any relevance 
After testing the speed of the motors it was deemed as irrelevant.
The motors moved with a speed that a target would have to move at a speed where the frame rate of the camera would be a bottleneck instead.
The speed at which the target had to move to solve the problems, would make prediction irrelevant, as the motors would catch up with no problem at all.


Due the lacking importance, and how this would add more development, this was deemed irrelevant and out of scope.

\subsection{Variable motor power}
Finding out how much power is needed when moving on a given axis.


An interesting result from the motor tests done in Section~\ref{des:sec:hwtest}, was that motors put under more stress depending on the current angle and wanted direction.
This was in part due to the additional weights of the structure, which creates a variable resistance on the $Y$-axis, and due to the wires that would create resistance when stretched, which was an issue when turning on the $X$-axis, due to the difficulty of cable managing imposed by LEGO.
The result was that the robot would not move at all in some cases, but could also result in the robot overshooting.
A solution is required.

This problem could be solved by creating a function that takes current degrees, direction and distance as input and outputs a required power.
This function could be found through either online or offline learning or through an design-time algorithm.
Design-time solutions could create inaccuracy if the device would be moved, or rewired, as the degree of resistance would change.
Online learning might not create a better solution, as the resistance shouldn't change drastically while the system is running, as it is not meant to be moved around.
Due to the above reasoning, the optimal solution should be created with offline learning.
The specific aspects of this is going to be covered in Section\todo{ref}