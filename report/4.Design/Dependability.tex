\section{Dependability}\label{Design:Dependability}
In most software systems dependability has to be accounted for, as discussed in Section~\ref{sec:dependability}.
This section will enlighten the thoughts behind the critical parts of the system, and try to enlighten the measures taken to ensure dependability.

\subsection{Fault Tolerance - Bit-flipping}
As a proof of concept, it was decided to implement bitflip fault tolerance in the laser module on the NXT, given the following rules.

\begin{enumerate}
  \item Every variable \texttt{x} must have a duplicated shadow variable \texttt{x0} \cite{errorDetectionSoft}
  \item Every write operation performed on \texttt{x}, must also be performed on \texttt{x0} \cite{errorDetectionSoft}
  \item After each read of \texttt{x}, consistency must be checked between \texttt{x} and \texttt{x0} \cite{errorDetectionSoft}
  \item For every conditional, repeat test in every conditional branch (both then and else branches) \cite{errorDetectionSoft}
\end{enumerate}

The code in Snippet~\ref{snip:without_bitflip} and Snippet~\ref{snip:with_bitflip} is written according to these four rules and ensures that bitflips have not happened for example when reading from variables or during evaluation of if-conditions.

\noindent\begin{minipage}{.45\textwidth}
\begin{lstlisting}[label={snip:without_bitflip},caption={Without~bitflip-security},frame=tlrb,numbers=none]
void toggle_laser() {
  if (!laser_state) {
    enable_laser();
  } else {
    disable_laser();
  }
}
\end{lstlisting}
\end{minipage}\hfill
\begin{minipage}{.45\textwidth}
\begin{lstlisting}[label={snip:with_bitflip},caption={With bitflip-security},frame=tlrb,numbers=none]
void toggle_laser() {
  if (!laser_state) {
    // Rule 4, conditional repetition
    if (laser_state) bitflip_error();
    // Rule 1+3, reading from variable.
    if (laser_state != laser_state_0) bitflip_error();

    enable_laser();
  } else {
    // Rule 4, conditional repetition
    if (!laser_state) bitflip_error();
    // Rule 1+3, reading from variable.
    if (laser_state != laser_state_0) bitflip_error();

    disable_laser();
  }
}
\end{lstlisting}
\end{minipage}

The primary reason for implementing bitflip protection in the laser module, is that it in a real life situation will be the most system critical component on the NXT.
Given that the laser module is responsible for firing when aiming at a target, a certain dependability and security is required to avoid accidental discharge, against non-targets.

For this reason, the proof-of-concept solution of bitflipping, has only been applied to the laser.
This implements \texttt{bitflip\_error()} protection by turning off the laser, and recovering the state of the laser module.
