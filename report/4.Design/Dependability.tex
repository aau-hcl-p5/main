\section{Dependability}\label{Design:Dependability}

\subsection{Fault Tolerance - Bit-flipping}
As a proof of concept, the group decided to implement bit-flip fault tolerance in the laser module on the NXT, given the following rules.

\begin{enumerate}
  \item Every variable \texttt{x} must be duplicated \texttt{x0} \cite{errorDetectionSoft}
  \item Every write operation perfomed on \texttt{x}, must also be performed on \texttt{x0} \cite{errorDetectionSoft}
  \item After each read of \texttt{x}, consistency must be checked between \texttt{x} and \texttt{x0} \cite{errorDetectionSoft}
  \item For every conditional, repeat test in every target (both then and else branches) \cite{errorDetectionSoft}
\end{enumerate}

Following the four rules above, th

\noindent\begin{minipage}{.45\textwidth}
\begin{lstlisting}[caption={Without bitflip-security},frame=tlrb,numbers=none]
void toggle_laser() {
  if (!laser_state) {
    enable_laser();
  } else {
    disable_laser();
  }
}
\end{lstlisting}
\end{minipage}\hfill
\begin{minipage}{.45\textwidth}
\begin{lstlisting}[caption={With bitflip-security},frame=tlrb,numbers=none]
void toggle_laser() {
  if (!laser_state) {
    if (laser_state) bitflip_error(); // Rule 4, conditional repetition
    if (laser_state != laser_state_0) bitflip_error(); // Rule 1+3, reading from variable.

    enable_laser();
  } else {
    if (!laser_state) bitflip_error(); // Rule 4, conditional repetition
    if (laser_state != laser_state_0) bitflip_error(); // Rule 1+3, reading from variable.

    disable_laser();
  }
}
\end{lstlisting}
\end{minipage}

The primary reason for choosing the laser-module, is that it in a real life sitation will be the most system critical component on the NXT.
Given that the laser will be the module, to be fired on target only when on target, to avoid accidental discharge.

For this reason,
