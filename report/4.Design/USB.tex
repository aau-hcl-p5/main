\section{USB communication}
\label{sec:usb}
As described in the component diagram in Section~\ref{sec:architecture}, the system will consist of two separate hardware platforms, that need to communicate.
As the NXT can only communicate with external devices using Bluetooth and USB, and we deemed Bluetooth too slow for our system, the group has decided to use USB.
According to the nxtOSEK documentation, the nxtOSEK library should also support built-in USB, through an easy to use API\cite{ecrobotUSB}.
When the group tested this functionality, it turned out that the nxtOSEK library for the computer, was no longer functional.

The group tried a few different solutions for custom NXT USB communication, but in the end, writing our own module to handle the communication was deemed the best solution.

The communication is described in the diagram here:

\figur{1}{images/USBCommunicationDiagram}{Diagram of the USB communication}{fig:compusb}

The USB communication module will consist of three parts.

The initialization part, which will perform a handshake with the NXT, initiating the USB receiving.
This part posed a bit of a challenge, as the handshake and its content was not documented, thus leaving the group in the dark about its existence.
The group used a USB sniffer, attached to a working compiled program, a usbtest sample, which is present in the samples directory when downloading the nxtOSEK library\cite{ecrobotUSB}.
This showed that the compiled program sent two bytes \texttt{x01}, a \texttt{1} and \texttt{xFF}, a \texttt{255}, before sending any additional data.
Based on our own testing of the USB capabilities, it was determined that these bytes signaled the NXT that USB communication would be commencing.

The data writing part will perform the actual communication with the device, and will be concerned with sending the actual packets of data.

Finally a disconnect part, which will send a message that signals the NXT to turn off the USB communication.



The implementation of the USB module will use the python library PyUSB, which is a python module, that allows for easy use of USB devices\cite{PyUSB}.
The group will implement a PyUSB module, which will handle the communication with the NXT.



