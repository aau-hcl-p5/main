\section{Real-time system}\label{Design:RTS} 
In order for the project to qualify as a real-time system it must be able to respond correctly in a timely manner.
In the following section, the design decisions made for the project in regards to ensuring this will be elaborated upon. 

\subsection{Scheduling}\label{Design:Scheduling}
When choosing a scheduling method for the system, multiple methods were considered. 
Both fixed priority scheduling and cyclic executive were considered for the system as both of them would be good at handling the tasks of the system with their deadlines.

Due to the fast execution time of the tasks, and that preemption and context switching would exceed the time for simply completing the previous task, it was at first deemed obvious to use the cyclic executive model as all tasks should run regularly, and that they execute fast enough to run within the major cycle the cyclic execution method should be perfect. 
However, should the execution time of a single task exceed one millisecond, the usage of cyclic executive is no longer viable, as the task to keep the USB connection alive is required to be executed once every millisecond.
In this case, fixed-priority scheduling is a better alternative, as it allows the scheduler to pre-empt the task in order to run the USB related task.

Both the fixed priority and cyclic executive methods are a common ways to implement hard real-time systems due to their simplicity, deterministic behavior and relative ease of implementation\cite{CyclicExecutionKimLarsen}.

In order to make sure that the system is schedulable an UPPAAL model of the scheduler and tasks where created. 

\subsubsection{UPPAAL}
UPPAAL is an integrated tool environment for modeling, validation and verification of real-time systems modeled as networks of timed automata\cite{UPPAALWebsite}. 
Inside the UPPAAL tool an model of the tasks in the system was made. 
On \autoref{fig:UppaalTask} the behaviour of the a task in can be seen. 
The model is a timed automaton which is used to ensure that the task will come true to its deadline and periods. 
The model consists of all the states a task can be in.
Each state has a number of transitions that can change the state of the task. 

\figur{0.5}{images/Task}{The figure shows the final base construction}{fig:UppaalTask}

On \autoref{fig:UppaalScheduler} the model of how the scheduler works can be seen. 
Since the scheduler is responsible for running all tasks on the system it simply needs an queue of tasks ready for running. 
It when simply needs to start the tasks as they arrive at a ready state. 
\figur{0.5}{images/Scheduler}{The figure shows the final base construction}{fig:UppaalScheduler}

Based on the model of both the task and the scheduler the system can be verified that it is schedulable using the verifier tool in UPPAAL. 
The verifier will run a query, as can be seen on \autoref{snip:UPPAALQuery} asking that for any given state, non of the tasks is going into the state error 
\begin{lstlisting}[label={snip:UPPAALQuery},caption={Query from UPPAAL verifier},frame=tlrb,numbers=none]
A[] not (BackgroundTask.Error or UpdateDisplayTask.Error or HandleLaserTask.Error or MoveMotorsTask.Error or ReciveDataTask.Error or KeepUSBAliveTask.Error)
\end{lstlisting}
The result of this query is a prof that the system is schedulable. 