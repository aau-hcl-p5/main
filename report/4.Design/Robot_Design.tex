\section{Robot Design}\label{Design:Robot}
In order to make a turret that could fulfil the requirements of the project an stable hardware platform for the system to operate on. 
The entirety of all physical components will here on simply be referred to as the robot.
Doing the duration of the project, the construction of the robot has changed several times, in the following subsections the different parts of the robot changes will be elaborated and the design criteria will be explained. 

\subsection{Design criteria}
Before the development of the robot can be conducted the design criteria from which the robot should be crafted from must be agreed upon. 
The agreed criteria is as follows:
\begin{itemize}
\item \textit{Robustness}
\item \textit{Simplicity}
\item \textit{Ease of access to components}
\item \textit{Correctness}
\end{itemize}

With \textit{robustness} meaning that the robot should be sturdy doing the movement of the robot.
\textit{Simplicity} meaning that the robot should be designed with as simple as possible to reduce the amount of work required from the project group. 
In order to make parts of the robot maintainable the criteria of ease of access to components is deemed essential. 
Finally the robot must be able to move correct according to the information delivered to the robot, hence making the correctness criterial essential. 

\subsection{The base support}
In order to achieve both \textit{robustness} and \textit{correctness} the robot should have a stable base of support. 
The base should tie in to all essential parts of the robot grounding them to a common fixed ground.
The base should be as wide as feasible to make the entire robot more stable. 

Since the project is based on the LEGO NXT and belonging motors the simple solution is to make the base with standard LEGO Technic parts made available by the university. 
\figur{1}{images/BaseFinal}{The figure shows the final base construction}{fig:RobotBaseFinal}

\subsection{The turret}
Once the base of the robot has been crafted the turret can be developed in order to achieve the design criteria of \textit{correctness} and \textit{robustness}. 

The turret has been through different iterations doing the development. 
On \autoref{fig:TurretEarly} the first iteration of the turret can be seen.
There where a multiple problems with this early design, first the top part of the turret where not stable enough for the robot to achieve the \textit{robustness}.
\figur{0.5}{images/TurretEarly}{The figure shows the first iteration of the turret design}{fig:TurretEarly}

After the first iteration an second attempt where made in order to correct the issues of the first. 
On \autoref{fig:TurretMiddle} first the top part of the turret turned out to be too heavy for the motor to handle. 
\figur{0.5}{images/TurretMiddle}{The figure shows the second iteration of the turret design}{fig:TurretMiddle}



Hence the next iteration of the turret where designed to be lighter. 
As can be seen on \autoref{fig:TurretFinal} the top of the turret has been significantly reduced in both complexity and weight. 
The most significant reduction on the turret the new laser mounted on the existing webcam which has been moved closer to the rotating axis hence reducing the power needed to move the turret. 
\figur{0.4}{images/TurretFinal}{The figure shows the final iteration of the turret design}{fig:TurretFinal}

\subsection{Cable management}
In order to make the robot according to the criteria of \textit{correctness} and \textit{robustness} the cables attached to the different components of the robot should have good cable management. 
The cable management should make the robot able to move without unplugging any cable.
However since the robot is constructed primarily using LEGO Technic, the cables are robust and hard to manage hence only the cables from the webcam and laser is manageable. 