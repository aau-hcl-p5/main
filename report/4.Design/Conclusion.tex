\section{Conclusion}
In the last few sections, different aspects of the system were considered.

\paragraph{Architecture}
A specific architecture was decided upon.
A graph of the different components was created, and based on this, the development of the solution can take shape concurrently.


\paragraph{Components}
It was reasoned that both an NXT and an external computer should be used.
The NXT will be utilized for the movement and predictions, whereas the computer is used for detecting the location of the target.


Motors and hardware as analyzed, tested and finally chosen.
These decisions will help shape the physical aspects of the solution.


\paragraph{Machine Intelligence}
It was determined that there were two obvious applications of MI in the project: object localization and variable motor power.

\subparagraph{Object localization}
Unfortunately, due to hardware limitations, no machine learned object localization algorithm will be fast enough to be feasible for the project.

Therefore, the fall back solution is to use the "Thresh Moment" algorithm, since it performs exceptionally well in terms of both performance and accuracy, while being very simple due to the utilization of the OpenCV functions.

However, if the hardware limitations were not an issue, the YOLO algorithm proved promising in terms of accuracy, and could be favored in future implementations, where the target is more complicated than simply "a red object".

\subparagraph{Variable motor power}
It was determined that the most obvious application of MI would be the implementation of variable motor power, as explored in Section~\todo{Ref to Thomas variable motor power section}
The reason was that the required power to move the motors a degree would vary greatly.
For this reason it was obvious to apply machine learning, as the \texttt{F.L.A.T} would be able to learn itself how much power to apply.
The specifics of this implementation will be described in Section~\todo{Ref to VMP implementation}.


Based on these considerations the actual solution can be implemented.
