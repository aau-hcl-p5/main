\section{Conclusion}
In the last few sections, different aspects of the system were considered.

\paragraph{Architecture}
A specific architecture was decided upon.
A graph of the different components was created, and based on this, the development of the solution can take shape concurrently.


\paragraph{Components}
It was reasoned that both an NXT and a Raspberry Pi should be used.
The NXT will be utilized for the movement and predictions, whereas the Raspberry Pi is used for detecting the location of the target.


Motors and hardware as analyzed, tested and finally chosen.
These decisions will help shape the physical aspects of the solution.


\paragraph{Object localization}
Unfortunately due to hardware limitations, no machine learning algorithm will be fast enough to be feasible in the project.
Therefore the fall back solution is to use the "Thresh Moment" algorithm, since it performs exceptionally well in terms of both performance and accuracy, while being very simple due to the utilization of the OpenCV functions.

However, if the hardware limitations were not an issue, the YOLO algorithm proved promising in terms of accuracy, and could be favored in future implementations, where the target is more complicated than simply "a red object".

The sum of these considerations should be a vision of a specific solution, which is creatable.
These considerations should be seen as the reasoning for the implemented solution expressed in the following chapter.
