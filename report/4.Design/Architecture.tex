\section{System Architecture}
\label{sec:architecture}
Before the implementation and building phases can begin, it is essential to give some thought towards the design of the system, both software and hardware wise.
\figur{1}{images/ComponentDiagram}{Component diagram for the system architecture}{fig:comparch}
\autoref{fig:comparch} shows the component diagram for the system.
The system will consist of two platforms: the LEGO NXT for handling the real time processing, and a Raspberry Pi for handling the machine intelligence.
Each platform is split into two groups: a hardware group containing the hardware which is connected to it and a software group containing a series of classes that must be implemented for the \texttt{F.L.A.T} to function.

Both the hardware and software requirements are defined with a high level of abstraction, since the diagram should be accurate regardless of implementation details or choice of hardware.

\subsection*{Dependencies}
The striped black arrows indicate the dependency between one or more elements of the chart.
For example, the method \textit{MoveTo} depends on the vertical and horizontal motors that are grouped together as \textit{motors} in order to function and are thus marked with a dependency arrow.
However, some software methods also depend on other methods to finish before they can execute, which can be seen in the software grouping in the Raspberry Pi.
Since the Raspberry Pi is responsible for the machine intelligence part of the system, the methods hereof mostly consist of image recognition, which has to happen in a given sequence:
\begin{enumerate}
\item Get data from camera
\item Run image recognition algorithm
\item Send result data to NXT
\end{enumerate}

Due to the dependencies of these methods, none of them can run independently without the result of the prior method, since it would simply lead to resending the same information twice.
Furthermore, it is important to notice that the diagram contains methods for both shooting a laser, and a projectile as mentioned in the requirements, \autoref{subsec:requirements}.
As the projectile firing model requires additional calculations, the device will function with merely the laser methods and hardware. 

With the architecture in place, it is now possible to focus on the details of what is required for implementing each part of the architecture.
In the following part, the details regarding the target information structure will be eluded upon.

\subsection{Target information}
A target information record has been defined internally to store relevant target information and a status code.
This data is calculated on the Raspberry Pi host and the record is then transferred onto the NXT, using the NXT's USB port.

The 4 status codes are defined with information about whether a target has been found, the device is on target, or whether the host has requested termination of the connection.

Furthermore, a location structure is kept with information regarding the $X$ and $Y$ distance relative to the target.
The $X$ and $Y$ values are based on the distance to the target from the center of the webcam image and the values are processed on the host such that they can be transformed into a value ranging from -127 to 127 as described in \autoref{sec:mappingcompvis}.
Likewise, the transformation function has the ability to do non-linear transformations, which can be used to increase the precision when the target is close to center, as opposed to when the distance is greater and less precision is needed.
