\section{System Architecture}
Before the implementation and building phases can begin, it is essential to give some thought towards the design of the system, both software and hardware wise.
In order to do so, a component architecture diagram was constructed, which can be seen on Figure~\ref{fig:comparch}.
\figur{1}{images/ComponentDiagram}{Component diagram for the system architecture}{fig:comparch}
As seen on the diagram, the system will consist of two platforms: the LEGO NXT for handling the real time processing, and a Raspberry Pi for handling the machine intelligence.
Each platform is split into two groups: a hardware group containing the hardware which is connected to it and a software group containing a series of classes that must be implemented for the device to function.
Both the hardware and software requirements are defined with a high level of abstraction, since the diagram should be accurate regardless of implementation details or choice of hardware.
 
\subsection*{Dependencies}
The striped black arrows indicate the dependency between one or more elements of the chart. 
For example, the method \textit{MoveTo} depends on the vertical and horizontal motors that are grouped together as \textit{motors} in order to function and are thus marked with a dependency arrow.
However, some software methods also depend on other methods to finish before they can execute, which can be seen in the software grouping in the Raspberry Pi.
Since the Raspberry Pi is responsible for the machine intelligence part of the system, the methods hereof mostly consist of image recognition, which has to happen in a given sequence:
\begin{enumerate}
\item Get data from camera
\item Run image recognition algorithm
\item Send result data to NXT
\end{enumerate}

Due to the dependencies of these methods, none of them can run independently without the result of the prior method, since it would simply lead to re-sending the same information twice.
Furthermore, it is important to notice that the diagram contains methods for both shooting a laser, and a projectile as mentioned in the requirements, Section~\ref{sec:moscow}.
%TODO lav et label ved moscow
As the projectile firing model requires additional calculations, the device will function with merely the laser methods and hardware. 
The methods required for projectile firing are colored red to indicate this.
With the architecture in place, it is now possible to focus on the details of what is required for implementing each part of the architecture.
In the following part, the details regarding [----] will be eluded upon.
%TODO Replace [----] with whatever part follows.