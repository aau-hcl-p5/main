\section{Performance}
In order to be able to track an object in real time, the object localization will have to happen in real time.
Therefore, the algorithms should be tested in order to determine whether they will be able to execute the object localization fast enough for it to be feasible for the project.

\subsection{Raspberry Pi}
The Raspberry Pi 3B is the desired host of the MI processing, but it is very limited in the amount of resources available to it.
It comes with a 1.2GHz Quadcore ARM Cortex-A53, 64Bit processor and 1 GB of RAM.
In relation to computation power, the Raspberry Pi is a little underwhelming compared to an average consumer computer.
Therefore, although an algorithm might execute fast enough on a normal computer, it is essential for it to be tested whether it will also be fast enough when executed on the Raspberry Pi.

\subsection{Object fill algorithm}
The object fill algorithm is limited by the framerate of the camera on an average to high end laptop, meaning that it could serve as a good baseline for further object localization algorithms.

The results of the conducted tests show that the Raspberry Pi show that it took approximately 0.07 seconds to analyze a frame.
This is roughly equivalent to an acceptable but not amazing frame rate of 15 frames per second.
However, it was significantly slowed down in the worst case scenario, where no object is to be found within the frame.
In this case, the algorithm was only able to process just below 10 frames per second.
Although this may seem troublesome, it is not much of a concern since the robot does not need to update once no object is within the frame, and the performance will increase as soon as the object is found.

\subsection{YOLO}

As YOLO is praised as one of the fastest object detection algorithms some tests were conducted in order to determine if it would be a viable option for the project.

The conducted benchmark tests used the fastest version of YOLO, called YOLO2 light along with a related lightweight model with a size of 33.8 MB on a modern high-end laptop, the Dell XPS 15.
The Dell XPS used in the test has a Quad-Core Intel I7 7700HQ with a base 2.8 GHz clock speed along with 16 GB of ram.

The tests conducted using YOLO2 light rendered impressive results regarding accuracy.
However, unfortunately the performance was no good as the XPS only managed to analyze around 2 frames per second in average.

There should be a couple of notes regarding these results.
First of all, the tests were conducted with YOLO utilizing the CPU.
Normally YOLO would utilize the GPU in order to achieve far better performance results, but as the Raspberry Pi will not be able to utilize a powerful GPU, testing on the GPU was deemed unnecessary. 
Another thing to note is the still fairly large model of 33.8 MB.
In the simple case of localizing a single object, this model could surely be decreased a lot in complexity, which in turn would render better performance results.

Furthermore, as previously mentioned, YOLO is an object detection algorithm.
This means the algorithm will try to locate and classify all objects in the image.
Performance could presumably be improved by only having a single class to classify and then terminate the algorithm as soon as the first object was localized.
However, it was decided that making modifications to YOLO would likely be too tedious, while risking that the modifications would still not make it fast enough to run on the Raspberry Pi.

In conclusion, it was determined that the YOLO algorithm would not be feasible for this project.

\subsection{Conclusion}
TODO ADD PERFORMANCE CONCLUSION - WHICH ALGORITHMS WAS CHOSEN BASED ON THESE PERFORMANCE TESTS. 
MAYBE CONCLUSION WILL BE THAT NO ML ALGORITHM IS FAST ENOUGH TO BE USED AND THEREFORE THE OBJECT FILL ALGORITHM WILL HAVE TO BE USED.