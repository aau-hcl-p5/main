\section{Object localization}\label{Design:ObjectLocalization}
An early working solution was prioritized in order to get a baseline product that could be further improved upon.
Therefore since object localization using machine learning algorithms requires a bit of setup, along with a lot of training data some algorithms were implemented without machine learning.
This section will elaborate on these algorithms.

\subsection{Zone Average}
Initially an object detection algorithm dubbed "Zone Average" was developed.
Zone Average locates the most red object in an image.
It works by splitting the image into $N\times N$ tiles, and determining the total redness within each tile.
The tile that contains the most red is determined to contain the object, and the location of that tile is returned as a result.

In order to increase accuracy, dynamic resizing of tiles were introduced.
The idea is that if the same tile has been the selected tile for multiple iterations, the size of the tiles should decrease to get better accuracy.
Likewise when the object begins moving the size of the tiles should expand.

\figur{0.6}{images/zone_avg.png}{Zone average algorithm detecting red balloon.}{fig:zone_avg}

Zone Average worked well as a starting point, but it lacked accuracy as also illustrated in image \ref{fig:zone_avg} where multiple tiles could be very red, but not necessarily in the center.

\subsection{Object Fill}\label{sec:objectfilldesign}
Following Zone Average a new algorithm called "Object Fill" was created.
Object Fills purpose was to render very accurate position results, with the initial limitation of only being able to locate a single coloured object.
The algorithm consists of two main parts:
\begin{itemize}
	\item Locate the object
	\item Fill the located object
\end{itemize}
The idea is to traverse the image with a given step size, until the last x pixels were within the given red threshold.
Once the location part yielded that a red object was found, a filling algorithm was initialized in order to determine the exact body of the red object.
The filling algorithm is in essence a graph search, and works as described below:
\begin{itemize}
	\item Create a queue with a pixel known to be within the red object (This pixel is reported by the initial object location algorithm).
	\item While queue is empty:
	\begin{itemize}
		\item Pop element from queue.
		\item If element is not within the red threshold
		\begin{itemize}
			\item Add the pixels x and y coordinate to the sum of x and y coordinates on the pixels bounding box.
			\item Increment the number of pixels on the objects bounding box with one.
		\end{itemize}
		\item Else
		\begin{itemize}
			\item Get all neighbors of the pixel (step size for neighbors may not necessarily be one) that has not yet been visited
			\item Add neighbors to the queue and to a set of all visited pixels
		\end{itemize}
	\end{itemize}
	Divide the sum of the bounding box's x and y values, with the total amount of elements in the bounding box to get the center x and y value of the object.
\end{itemize}

Figure~\ref{fig:obj_fill} is a debug view of the algorithm running.
The Green pixels are the bounding box of the object and the black pixels shows all the visited pixels.

\figur{0.6}{images/obj_fill.png}{Object Fill algorithm detecting red balloon.}{fig:obj_fill}

The algorithm was later improved in a couple of aspects.
First, the initial search for the red object would start its search based on the last center of the red object.
Secondly, the algorithms was improved to only accept an object if it contained a certain amount of "redness".
This helped reduce the amount of false positives from very small objects or big but not so red objects.
If the filling algorithm reports that the object is not a valid red object, the searching algorithm will continue from where it left off, until it has searched the entire image.

\subsection{Thresh Moment}

The previous algorithms utilizes OpenCV to extract images from the webcam. 
The algorithms then work with the data in a low level fashion iterating over each pixel in the image to analyze it.
However, OpenCV comes with a huge amount of functions for image processing, that can be utilized in order to create a simple performant algorithm for this use case. 
The algorithm was dubbed "Thresh Moment" since the core algorithm relies on threshold and image moment.

A couple of concepts used by the algorithm are worth mentioning beforehand.

\paragraph{HSV}
The algorithm will convert the image to the HSV color model.
HSV format differs from traditional RGB, in that instead of having three additive color channels for red, green and blue it has three values defining hue, saturation and light.
This makes it easier to recognize both dark and light red, since the redness is defined in the hue.

\paragraph{Contours}
Contours are almost identical to edges.
They are outlines that follows all continuous points along a boundary of an object.
Continuous points are determined based on color and intensity.
In this case it will essentially help determine the outline of the red object.\cite{contours}

\paragraph{Image moment}
The moment is the weighted average of an objects pixel intensities.
It can be used to determine the area of the object and the average x and y coordinates, also called the centroid or the center of the object.


The algorithm consists of the following steps:
\begin{itemize}
	\item Convert frame to HSV
	\item Mask all pixels within the HSV threshold range of red.
	The mask is essentially a binary image, where all pixels within the range is white, and everything else is black.
	\item Find contours in the image. 
	This will be the outline of the red object.
	\item Filter all contours less than a given size, to remove noise.
	\item Find the greatest remaining contour.
	\item Find the image moment of the contour.
	\item Using the moment, determine the center position of the object.
\end{itemize}

A debug view of the algorithm can be seen in figure~\ref{fig:thresh_moment}.
The white part is the mask and the blue outline is the contour.

\figur{0.8}{images/thresh_moment.png}{Thresh Moment algorithm detecting red balloon.}{fig:thresh_moment}

\subsection{YOLO}

The YOLO algorithm itself is explained in the analysis section~\ref{sec:YOLOAnalysis}.
YOLO can be run using darknet, which is an open source neural network framework written in C and CUDA\cite{darknet13}.
It is also easily integrated into your own project, and need not be executed as a standalone program.
In figure~\ref{fig:yolo_example} the YOLO2 Light version is executed with a simple model, that can recognize a small set of everyday items, like for instance a person.

\figur{0.6}{images/yolo_example.png}{YOLO algorithm detecting person.}{fig:yolo_example}

\subsection{Scikit classifier}
\todo{Fill this if we are able to create a proper classifier with scikit}
