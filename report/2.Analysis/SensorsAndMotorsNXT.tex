\section{Sensors and motors for Lego NXT}
\label{sec:sensors}
Based on the usage of the Lego NXT series as the chosen platform, it is essential to research the functionalities of the components. 
The NXT series offers a large variety of sensors and motors available for use, ranging from light and sound sensors to dual range, triple zone infrared obstacle detectors.
Due to the amount of available components, it is pertinent to further investigate the advantages and disadvantages of each component to find the best applicable ones for the project.\\
In this section, the focus will be held on the basic functionality of the available components.

\todo{GOT TO HERE}

\subsection{Ultrasonic Sensor}
The primary purpose of an ultrasonic sensor is to determine distance from the origin to one or more objects using ultrasonic waves.
To do this, the sensor head sends out an ultrasonic wave and receives the wave reflected back after hitting the surface of the target object.
Using the time interval elapsed between sending and receiving the wave, the distance can be calculated.\\
The NXT Ultrasonic Sensor has an effective range of 1 to 250 centimeters with an accuracy of +/- 1 centimeter.

\subsection*{Touch sensor}
As implied in the name, this type of sensor allows detection of touch.
This allows for a variety of functionality such as enabling the machine detect if it collides with other objects.

\subsection*{Light sensor}
The light sensor allows measuring the light levels by using an LED and a photo resistor.
The sensor is capable of both measuring ambient light, such as the light levels of a room, and reflected light by using the built-in LED.

\subsection*{Dual range, triple zone infrared obstacle detector}
The dual range, triple zone infrared obstacle detector, also known as NXTSumoEyes, is an obstacle sensor with an effective range of up to 20 centimeters using infrared beams.
The major usability difference between the NXTSumoEyes and the ultrasonic sensor is that the NXTSumoEyes works with three individual detection zones of 20 degrees each, allowing for a more detailed response on the location of the detected object.

\subsection*{IRRD-T sensor}
The infrared relative distance sensor, IRRD, measures the distance to an object with an effective range of up to 60 centimeters by measuring reflected infrared light, in a similar faction to the ultrasonic sensor.

\subsection*{HPMR Infrared Distance Sensor}
The high precision medium range infrared distance sensor work in the same way as the IRRD-T sensor previously described.\\
However, this sensor allows detection of objects with a range of 10 to 80 centimeters, by using a Sharp GP20A02YK sensor, which is slightly more powerful than the sensor used in the IRRD-T sensor.

\subsection*{Motors}
The NXT series offers a variety of motors in two categories: DC motors and servo motors.
The DC motors are regular powered motors that can simply be turned on and off, while the servo motors have the functionality of controlling the rotation by a given degree.\\
The servo motors have a speed of 160-170 RPM, and a rotation precision of 1 degree.
In comparison, the DC motor has a maximum speed of 380 RPM, making it faster but allowing less control of the rotation.

\subsection*{Angle sensor}
The angle sensor measures axle rotation position and rotation speed.
It allows measuring both the absolute angle, meaning the rotation of an axle from 0 to 359 degrees with an accuracy of 1 degree.\\
Likewise, it can measure an accumulated angle, which is the accumulated multiple rotation angle measured since it was last reset.
Finally the angle sensor can calculate the current RPM with a rotation rate from 1 to 1000 RPM, with an accuracy of 1 RPM.

As seen in this section, the NXT platform offers a large variety of sensors that can be useful for the construction of the device.
In this project the most relevant hardware elements are going to be motors, for turning  the device, and a sensor for detecting where the target is. 
