\section{Object detection, localization and tracking}\label{sec:anal:objdet}
Before examining the different ways to look for objects using computer vision, it is important to ensure a correct understanding of the terminology used when working with objects in images.
Object recognition is the discipline of computers working with objects in images.
Object detection, localization and tracking are all subtasks of object recognition.
\todo{can the previous two lines be combined to one?}

The difference between localization, detection, and tracking is subtle, yet essential\cite{objecttrackdetect}.
\begin{description}
    \item[Object classification:] Identifying objects in an image.
    \item[Object localization:] Finding the location of a single object in an image{.} Here it is irrelevant whether it is a bounding box for the object, an actual outline, or just an average location. 
    \item[Object detection:] Finding the location of multiple objects in an image{.} Usually the location of each individual object will be outlined with a bounding box, to ensure clarity as to which pixels represent each object.
    \item[Object tracking:] Doing object localization or detection over time, and optionally storing the previous locations.
    \item[Object recognition:] Finding objects in an image, regardless of it being detection, localization, tracking or classification. 
\end{description}