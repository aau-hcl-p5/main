\section{Object detection, localization and tracking}\label{sec:anal:objdet}
Before examining the different ways to look for objects using computer vision, it is important to ensure the correct understanding of the terminology used when working with objects in images.
%Dobbelt "sætning", men kan nok ikke fixes.
Object recognition is the discipline of computers working with objects in images, and object detection, localization and tracking, are all subtasks of object recognition.
The difference between detection, localization and tracking is subtle, however it is necessary to understand the difference\cite{objecttrackdetect}.

A simple aspect of computer vision is object classification, which is the identification of objects in an image.

\paragraph{Object localization:} 
Determining the location of a single object in an image. 
Here it is irrelevant whether it is a bounding box for the object, an actual outline, or just an average location.

\paragraph{Object detection:} 
Determining the locations of multiple objects in an image.
Usually the location of each individual object will be outlined with a bounding box, to ensure clarity as to which pixels represent each object.

\paragraph{Object tracking:} 
Doing either object localization or object detection over time and storing the previous locations.

In the rest of the report, object tracking will mean to find the location of an object in a sequence of images.
Object localization will refer to finding the location of a single object in an image.
Object detection will refer to finding the location of multiple objects in an image.
Finally object recognition is a general term, meaning finding objects in an image, regardless of it being detection, localization, tracking or classification. 
