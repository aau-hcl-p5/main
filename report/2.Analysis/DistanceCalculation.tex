\section{Firing at a moving target}
Hitting a moving target with a projectile is challenging, as the speed of both the target and projectile needs to be considered.

\subsubsection{Hitting a moving target with a projectile}
As the projectile requires time to travel, a target in motion will have moved during the time it took the projectile to reach the targets original location, meaning that the projectile has to be aimed at where the target will be, not where it currently is.


To solve this, an intersection point between the trajectory of the target, and the trajectory of a projectile, fired from the device, needs to be found.

The intersection point changes, depending on the speed of both the target and the projectile, and the distance between the target and the device, as a shorter distance equals less travel time.
This point is based on the speed of both, and the distance between the target and the device. 
The higher velocity of the projectile, the closer to the target the intersection point is going to be.

Based on the visual feed, as described in \autoref{sec:obj_tracking}, the predicted movement of the target should be calculable.


This still leaves the problem of determining the distance between a target and the device.
\subsection{Distance to target}
Finding the distance to a target is doable, and several solutions have already been explored.
As presented in \autoref{sec:sensors}, an \textit{ultrasonic sensor} mounted to the camera could be used.

\todo{det her er nok forkert, og det er ikke dyrt, som thomas har sagt, vi kunne bruge to webcams}
A more expensive solution would be using a 3D camera, but as these are relatively expensive and would further complicate the project, this option will not be explored any further.

However, even when knowing the distance, predicting where to shoot the target would be a complex task.
The trajectory of a fired projectile is highly dependent on the type of projectile.
An actual gun would shoot bullets at a high velocity and reliably, however using a gun in a university project is not feasible, and a LEGO projectile would most likely be unpredictable, as they are only meant as a toy, and are not designed for consistent trajectories.


As the goal of the project is to explore machine intelligence in a real-time embedded system, adding the prediction of trajectories and the firing of projectiles adds a lot of complexity to an already complex project.

An alternative to firing a projectile at an object would be to point a laser at the target and toggle it on when it is on the target.
Determining the distance between the device and the target is only required if there is a travel time on the projectile that is being fired at the target.
A laser will move at the speed of light, rendering the distance variable irrelevant.


\subsection{Takeaways}\label{anal:laser:takeaway}
To reduce the complexity of the solution, and be able to reach the project deadline, it makes sense to use a laser initially. 

Therefore the initial goal will be to hit the target with a laser.

Depending on the time used in developing this initial solution, the scope might be broadened to include an actual projectile.\todo{Might?}
