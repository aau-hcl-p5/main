\section{Structure of Codebase}
The code for the solution is split into two parts.
One part is dedicated to locating the object, while the other part handles the movement of the robot.
The location module is written in python 3{.}7 and is intended to run on an external computer.

The movement processing runs on an NXT, running nxtOSEK, and is written in C, version C90 with GNU-extensions.

% \subsection{Movement}
% The codebase of the NXT is separated into different files with distinct responsibilities, resulting in modularity, much like described above.

% \begin{itemize}
% 	\item \texttt{tasks}
% 	\item \texttt{usb}
% 	\item \texttt{display}
% 	\item \texttt{movement}
% 	\item \texttt{calibration}
% \end{itemize}

% The \texttt{tasks.c} file is the entry point of the project, which creates all the tasks that should be run on the system.
% \todo{Skriv om tasks.c} 
