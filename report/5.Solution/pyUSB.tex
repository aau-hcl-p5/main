\subsection{Python USB implementation}\label{sol:subsec:pythonusb}
%Denne er måske alt for detaljeret
In \autoref{sec:usbdes} the overall design of the USB communication was described.
The following subsection will describe how the USB communication is implemented in the Python part of the the solution.
The NXT specific USB implementation will be described in \autoref{sec:nxtusbimp}.

The USB device inherits from the OutputDevice base class.
The implementation uses the PyUSB library, as described in \autoref{sec:usbdes}\cite{PyUSB}.

\autoref{lst:PythonUSBInit} shows the initialization of the NxtUsb module, which handles the USB communication from Python to the NXT.
\begin{lstlisting}[label={lst:PythonUSBInit},caption={The initialization of PyUSB{.} Comments removed}]
    def __enter__(self):
        self.device = usb.core.find(idVendor=ID_VENDOR_LEGO, idProduct=ID_PRODUCT_NXT)

        if self.device is None:
            raise DeviceNotFound('Device not found')

        self.device.set_configuration()

        self.out_endpoint, self.in_endpoint = self.device[0][(0, 0)]
        self.out_endpoint.write(b'\x01\xFF')
        self.device.read(self.in_endpoint.bEndpointAddress, 8)
        self.initialized = True
        return self
\end{lstlisting}

The code itself is relatively self-explanatory, and only the most interesting parts will be elaborated upon.
Line 9 sets the endpoints, meaning where to read from and where to write to.
%\todo{Verify this}% altid samme endpoint i USB. Det er out; default write.
The same endpoint is used for both.

At this point in the program, the NXT is waiting for a specific code, which is sent in line 10.

When the NXT receives the code, it returns \texttt{{.}ecrobot}.
Should that be the case, it sets its initialized field to \texttt{True} and returns.



% Line 2 simply finds a USB device that matches the signature of the NXT.
% \todo{Kan ikke huske om næste linie er sandt}
% The signature was found using device manager with the NXT plugged in.
% In the case there is no NXT device found, an exception is raised, as shown in line 4.

% Line 7 sets the configuration of the USB device.
% This part is PyUSB specific, however, when the method is called without arguments, the current configuration is used.
% The specifics of what a configuration is do not matter for this specific use case.


With the device initialized, communication can now happen.
The communication itself is relatively simple, with the read method reading directly from the byte stream, as shown in \autoref{lst:PythonUSBread}.

\begin{lstlisting}[label={lst:PythonUSBread},caption={Reading from the USB port connected to the NXT}]
def read(self) -> bytes:
    return self.device.read(self.in_endpoint.bEndpointAddress, 8)
\end{lstlisting}

When writing to the device, there are two options:
The first, \texttt{write\_location}, writes a location to the USB, while the second, \texttt{write\_status}, writes a status.
Both are shown in \autoref{lst:PythonUSBwrite}.

\begin{lstlisting}[label={lst:PythonUSBwrite},caption={Writing to the USB port connected to the NXT}]
def write_location(self, data: Vector) -> None:
    self.out_endpoint.write(bytes([
        0,
        0,
        int(data.x) & 0xFF,
        int(data.y) & 0xFF
    ]))

def write_status(self, status: Status):
    value = status.value
    if type(value) is tuple:
        value = value[0]
    self.out_endpoint.write(bytes([
        int(value) & 0xFF,
        0
    ]))
\end{lstlisting}

Both methods send byte wise data.
