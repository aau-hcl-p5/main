\section{Calibration}\label{sec:calibration}

As the device required varying power in order to move, depending on the current revolution of the motor, the first goal was to extract these powers.
In order to do so, a calibration function was implemented that will slowly increase the power on the motor, until it observes a change in revolution.
A simplified version of the function can be seen in \autoref{lst:GetRequiredPower}.


\begin{lstlisting}[language=C,label={lst:GetRequiredPower},caption={Getting required power to move }]
int8_t get_power_to_move(T_AXIS_TYPE axis, T_DIRECTION direction) {
	int8_t power = MIN_POWER;
	T_REVOLUTION first_revolution = get_current_revolution();

	do {
		set_motor_speed(axis, power * (direction == POSITIVE ? 1 : -1));
		power++;

		// wait 30ms but make sure we don't move in that timeframe
		for(int i = 0; i < 30; i++)
		{
			systick_wait_ms(1);
			if (!should_stop_moving(first_revolution, power))
				break;
		}

	} while(should_stop_moving(first_revolution, power));

	set_motor_speed(axis, 0);

	return power;
}

\end{lstlisting}

The \texttt{should\_stop\_moving} function will determine whether the motor has moved, or if the power exceeds the maximum allowed power, meaning it is unable to move further.
The data is sent back to a host machine over the established USB connection.
A plot of the data for moving the motor from the lowest possible revolution to the highest can be seen in \autoref{fig:calib_data}.

\figur{1}{images/calib_data.png}{Required power for revolution on motor}{fig:calib_data}

The data is, as seen on the figure, jagged, which may be a result of hardware insecurities.
The goal is to find a function that approximates the data.
For this purpose, a neural network was trained on the data using scikit-learn for Python.
\autoref{lst:mlpregressor} shows how to train a multi layered neural network for regression.

\begin{lstlisting}[language=python,label={lst:mlpregressor},caption={Training a MLPRegressor with scikit}]
model = MLPRegressor(solver="lbfgs", activation="logistic", hidden_layer_sizes=(30, 30))

regressor.fit(input_values, expected_values)
\end{lstlisting}

The \texttt{MLPRegressor} has a lot of parameters that can be modified in order to tweak the neural network for your use case.
In this case default values are used for everything except solver, activation and size of hidden layers.
The activation function is set to use the logistic sigmoid function, and the model is defined to have 2 hidden layers with 30 neurons in each.
These values are obtained by trying out a lot of different values, in order to determine which values generally perform well on the data set.

\figur{1}{images/calib_data_nn.png}{Sample fitting of the data}{fig:calib_data}

Now this model has to be exported to the NXT.
In order to do so, some C code is generated based on the weights and biases from the model.
The C code contains all the weights and biases along with the activation function and a function to calculate the output given input.

The generated model is big, but a simplified, condensed version can be seen in \autoref{lst:exported_model}:
\todo{big how}

\begin{lstlisting}[language=C,label={lst:exported_model},caption={Autogenerated model for getting power to move up}]
typedef struct {
	double input_0;
} T_MODEL_INPUT;

typedef struct {
	double output_0;
} T_MODEL_EXECUTION_RESULT;

// Approximated sigmoid without exp, since not working on NXT
static double sigmoid(double value)
{
	double x = value >= 0 ? value : value * -1;
	double x2 = x * x;
	double e = 1.0f + x + x2 * 0.555f + x2 * x2 * 0.143f;
	return 1.0f / (1.0f + (value > 0 ? 1.0f / e : e));
}
static double WEIGHTS_LAYER_0[1][30] = {
	{ -0.2754047714229132, ..., 0.28265923487311884 }
};
static double WEIGHTS_LAYER_1[30][30] = {
	{ 0.337562467866465, ..., -7.977009995173775 },
	...
	{ -0.3327450268753057, ..., -0.5111629590067429 }
};
static double WEIGHTS_LAYER_2[30][1] = {
	{ 1.0582144881157733 },
	...
	{ 8.686207256470313 }
};
static double BIAS_LAYER_0[30] = {
	-2.163581253001705, ..., 0.4272619624684243
};
static double BIAS_LAYER_1[30] = {
	-0.03394434965821748, ..., -0.9310096193979974
};
static double BIAS_LAYER_2[1] = {
	-3.0451938474457716
};
T_MODEL_EXECUTION_RESULT calculate_model_up(T_MODEL_INPUT input) {
	// Calculate result
}

\end{lstlisting}

The calculate model function will utilize the calculations explained in section~\todo{Add ref to calculating in neural network} to calculate the values.
