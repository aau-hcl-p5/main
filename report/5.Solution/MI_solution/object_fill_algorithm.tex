\subsubsection{Object Fill algorithm}\label{solution:objfillalgo}

The core of the Object Fill algorithm is the actual \textit{filling} part.
It can be seen in a stripped down version in \autoref{lst:objectfill}.

\begin{lstlisting}[language=Python,label={lst:objectfill},caption={Stripped down version of object fill from object\_fill.py}]
	def _fill_get_center(self, object_position: Vector, frame: np.ndarray, image_size: Vector) -> Optional[Tuple[Vector,bool]]:
		queue: Deque = deque()
		queue.append(object_position)
		self._blacklisted_pixels.add(object_position)
		sum_redness = 0
		# used to calculate whether the bounding box covers the center
		outline = []
		
		while queue:
			element = queue.popleft()
			pixel_redness = _redness(element.x, element.y, frame)
			# is a bounding pixel
			if pixel_redness < self.red_threshold:
				outline.append(element)
				continue
			
			for neighbour in self._get_neighbours(element, image_size) - self._blacklisted_pixels:
				sum_redness += pixel_redness
				self._blacklisted_pixels.add(neighbour)
				queue.append(neighbour)
		
		sum_outline = Vector(0, 0)
		for e in outline:
			sum_outline += e
		
		if sum_redness > DEFAULT_MIN_TOTAL_REDNESS:
			on_target = any(
				math.sqrt((p.x - self.center.x) ** 2 + (p.y - self.center.y) ** 2) < self.fill_step_size
				for p in self._blacklisted_pixels
			)
			return sum_outline / len(outline), on_target
		else:
			return None

\end{lstlisting}

A high level explanation of the algorithm can be found in \autoref{sec:objectfilldesign}.