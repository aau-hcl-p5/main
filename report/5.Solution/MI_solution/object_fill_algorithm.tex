\subsection{Object Fill algorithm}\label{solution:objfillalgo}

The core of the Object Fill algorithm is the actual "filling" part.
It can be seen in a stripped down version in listing~\ref{lst:objectfill}

\begin{lstlisting}[language=Python,label={lst:objectfill},caption={Stripped down version of object fill from object\_fill.py}]
    def _fill_get_center(self, object_position: Vector, frame: np.ndarray, image_size: Vector) -> Optional[Vector]:
	    queue: Deque = deque()
	    queue.append(object_position)
	    self._blacklisted_pixels.add(object_position)
	    sum_outline = Vector(0, 0)
	    sum_redness = 0
	    sum_elements_in_outline = 0
	    
	    while queue:
		    element = queue.popleft()
		    pixel_redness = _redness(element.x, element.y, frame)
		    # is a bounding pixel
		    if pixel_redness < self.red_threshold:
			    sum_outline += element
			    sum_elements_in_outline += 1
		    
		    for neighbour in self._get_neighbours(element, image_size) - self._blacklisted_pixels:
			    sum_redness += pixel_redness
			    self._blacklisted_pixels.add(neighbour)
			    queue.append(neighbour)
	    
	    if sum_redness > DEFAULT_MIN_TOTAL_REDNESS:
		    return sum_outline / sum_elements_in_outline
	    else:
		    return None
\end{lstlisting}

A high level explanation of the algorithm can be found in Section~\ref{sec:objectfilldesign}.