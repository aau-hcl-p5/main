\subsubsection{Thresh Moment Algorithm}\label{solution:thresh_moment}
A stripped down version of the Thresh Moment algorithm can be seen in the following \autoref{lst:threshmoment}, with inline comments.

\begin{lstlisting}[language=Python,label={lst:threshmoment},caption={Stripped down version of thresh moment from thresh\_moment.py}]
	def locate_center(self, frame: np.ndarray) -> Optional[Vector]:
	    # Convert image to HSV
	    hsv_frame = cvtColor(frame, COLOR_BGR2HSV)
	    # Get mask from red threshold
	    mask = inRange(hsv_frame, (0, 150, 50), (10, 255, 255)) inRange(hsv_frame, (170, 150, 50), (180, 255, 255))
	    
	    # Find contours from mask
	    _, contours, _ = findContours(mask, RETR_TREE, CHAIN_APPROX_SIMPLE)
	    # Filter contours with areas that are considered too small to be the target
	    contours = [contour for contour in contours if contourArea(contour) > 20]
	    
	    if len(contours) != 0:
		    biggest_contour = max(contours, key=contourArea)
		    
		    # Find moment from the biggest contour
		    moment = moments(biggest_contour)
		    # Get center x and center y coordinate based on moment
		    cx = int(moment["m10"] / moment["m00"])
		    cy = int(moment["m01"] / moment["m00"])
		    return Vector(cx, cy)
	    return None
\end{lstlisting}

A general detailed explanation of thresh moment was described in \autoref{des:thresh}.

The algorithm gets an image as an array, and converts it to HSV.
Based on the threshold for the red color, the areas in the image within the red color range are found. 
Areas, or contours, that are too small, when compared to the other contours, are filtered out.
If there are still contours, the largest contour is selected, and the center of the selected contour, is returned as the location.