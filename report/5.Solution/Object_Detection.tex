\subsection{Object Localization}\label{solution:ObjectLocalization}
As mentioned in \autoref{flatcontrollerimplementation} there are multiple implemented options for object localization algorithms.
The object localization has to return the direction in which the target is, and whether target is covering the center of the screen.
This is used for determining whether the laser should be turned on or off.
This subsection will describe the two primary algorithms, that were used throughout the development of the project.
They all implement the \texttt{ObjectLocalizer} interface, which can be seen in \autoref{lst:objectlocalizer}.

\begin{lstlisting}[language=Python,label={lst:objectlocalizer},caption={The interface of all object localization algorithms}] 
class ObjectLocalizer:
	def locate_center(self, frame: np.ndarray) -> Optional[Tuple[Vector, bool]]:
		raise NotImplementedError
\end{lstlisting}
This interface exposes a single function, namely the \texttt{locate\_center} function, taking an image as input and returning a nullable tuple, including a vector, the target direction, and a bool stating whether the target covers the center.
\subsection{Object Fill algorithm}\label{solution:objfillalgo}

The core of the Object Fill algorithm is the actual "filling" part.
It can be seen in a stripped down version in listing~\ref{lst:objectfill}

\begin{lstlisting}[language=Python,label={lst:objectfill},caption={Stripped version of thresh moment from thresh\_moment.py}]
    def _fill_get_center(self, object_position: Vector, frame: np.ndarray, image_size: Vector) -> Optional[Vector]:
	    queue: Deque = deque()
	    queue.append(object_position)
	    self._blacklisted_pixels.add(object_position)
	    sum_outline = Vector(0, 0)
	    sum_redness = 0
	    sum_elements_in_outline = 0
	    
	    while queue:
		    element = queue.popleft()
		    pixel_redness = _redness(element.x, element.y, frame)
		    # is a bounding pixel
		    if pixel_redness < self.red_threshold:
			    sum_outline += element
			    sum_elements_in_outline += 1
		    
		    for neighbour in self._get_neighbours(element, image_size) - self._blacklisted_pixels:
			    sum_redness += pixel_redness
			    self._blacklisted_pixels.add(neighbour)
			    queue.append(neighbour)
	    
	    if sum_redness > DEFAULT_MIN_TOTAL_REDNESS:
		    return sum_outline / sum_elements_in_outline
	    else:
		    return None
\end{lstlisting}

A high level explanation of the algorithm can be found in Section~\ref{sec:objectfilldesign}.
\subsection{Thresh Moment Algorithm}\label{solution:thresh_moment}

A stripped down version of the Thresh Moment algorithm can be seen in listing~\ref{lst:threshmoment}, with inline comments.

\begin{lstlisting}[language=Python,label={lst:threshmoment},caption={Stripped version of thresh moment from thresh\_moment.py}]
	def locate_center(self, frame: np.ndarray) -> Optional[Vector]:
	    # Convert image to HSV
	    hsv_frame = cvtColor(frame, COLOR_BGR2HSV)
	    # Get mask from red threshold
	    mask = inRange(hsv_frame, (0, 150, 50), (10, 255, 255)) inRange(hsv_frame, (170, 150, 50), (180, 255, 255))
	    
	    # Find contours from mask
	    _, contours, _ = findContours(mask, RETR_TREE, CHAIN_APPROX_SIMPLE)
	    # Filter contours with areas that are considered too small to be the target
	    contours = [contour for contour in contours if contourArea(contour) > 20]
	    
	    if len(contours) != 0:
		    biggest_contour = max(contours, key=contourArea)
		    
		    # Find moment from the biggest contour
		    moment = moments(biggest_contour)
		    # Get center x and center y coordinate based on moment
		    cx = int(moment["m10"] / moment["m00"])
		    cy = int(moment["m01"] / moment["m00"])
		    return Vector(cx, cy)
	    return None
\end{lstlisting}
