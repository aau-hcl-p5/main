\section{Build process}\label{sec:buildprocess}
The build process for the project is quite complex as it targets the NXT.
This complicates the build process for a number of reasons:
nxtOSEK is the operating system for RTS on the NXT, and even though it is quite old, the latest release, as of the writing of this report, was released in January of 2013\cite{osekrelease}.
Additionally the nxtOSEK build process has a number of additional requirements.

During the groups testing of the nxtOSEK build, it was found that it was largely incompatible with modern operating systems, both Windows, OS X and a number of modern Linux distributions.
According to the latest documentation on the nxtOSEK webpage, the latest supported version of Linux is Ubuntu 11.10 which was released in 2011.
This complicated things a bit, as none of the group members were interested in installing such an old operating system on their personal computer, and instead Docker was used to avoid this.

\subsection{Docker}\label{subsec:docker}
Docker is a way to deploy a consistent environment for applications to run in, also known as containerization\cite{dockerdocstart}.
A Docker deployment consists of three different concepts, the Docker file, the container and an image.
\begin{description}
    \item [Dockerfile], which is a definition of a Docker image.
    \item [Images] are executables that includes all the required information to instantiate containers.
    \item[Containers] are the runtime instances of an image, the running process of an image.
\end{description}

A Dockerfile simply consists of lines, that describe how to build the image.
Each line applies a layer to the Docker image, and each of these layers are cached.
Thus if line 5 in the Dockerfile is changed, everything below line 5 is rebuilt.

The Dockerfile for this project is shown below, in Subsection~\ref{subsec:dockerimplementation}.

The Docker image is simply an executable, that instantiate a Docker container.

A Docker container is in broad strokes the same as a virtual machine, although a Docker container only uses the required resources, while a virtual machine is allowed access to more resources than is typically needed.
Docker runs natively on Linux, and takes no more memory than other executables.

The concept of containerization as a way to ensure consistency in deployment of software, was perfect for the purposes of the group.
Docker is supported on all 3 major platform, and while both Docker for Windows and MacOS run in a virtual machine, this is a non-issue.

A potential issue was nxtOSEK not being officially supported past Ubuntu 11.10, and Docker officially supporting Ubuntu 18.04, 16.04 and 14.04 \cite{dockerubuntu}.

This proved to be a non-issue, as modifying dependencies, and some tweaking, nxtOSEK ended up working fine on Ubuntu 14.04.

\subsection{Docker implementation}\label{subsec:dockerimplementation}
The actual Docker build process is somewhat obscure.

The first step is relatively straightforward with \textit{FROM ubuntu:14.04} simply specifying the base ubuntu image to use.
The next bit simply installs the packages required by nxtOSEK.
\begin{lstlisting}[language=docker,label={lst:dockerimplementation1},caption={Version definition and installation of packages required by nxtOSEK}]
FROM ubuntu:14.04

# Install required packages listed by nxtOSEK
RUN apt-get update
RUN apt-get -y install tk-dev ncurses-dev libmpfr-dev wget gzip tar software-properties-common xvfb
\end{lstlisting}

The next part is the most obscure part, as wine is installed to be able to run Windows programs.
This is done as the tool to build OIL files, only exists for Windows.
NeXT Tools is an utility program that includes various utilities for use with nxtOSEK, including an enhanced OIL file compile\cite{nxttool}.
It also includes an additional nxtOSEK requirement, texinfo, which is the official GNU documentation format\cite{texinfo}.

The version of texinfo on the apt repository was not compatible, hence why a specific version was downloaded.

\begin{lstlisting}[language=docker,label={lst:dockerimplementation2},caption={Installation of Wine and texinfo}]
# Install wine
RUN dpkg --add-architecture i386
RUN wget -nc https://dl.winehq.org/wine-builds/Release.key
RUN apt-key add Release.key
RUN apt-add-repository https://dl.winehq.org/wine-builds/ubuntu/
RUN apt-get -y install apt-transport-https
RUN apt-get update
RUN apt-get -y install --install-recommends winehq-staging

# Download a specific version of texinfo, the one in 14.04
# sources is not compatible with the expected version
RUN wget http://ftp.gnu.org/gnu/texinfo/texinfo-4.13.tar.gz
RUN gzip -dc < texinfo-4.13.tar.gz tar -xf -
RUN cd texinfo-4.13 \&\& ./configure \&\& make \&\& make install
\end{lstlisting}
Next, the ARM tool chain is installed.
The tool chain is used by nxtOSEK to build for the ARM7 CPU in the NXT.
\begin{lstlisting}[language=docker,label={lst:dockerimplementation3},caption={Building the ARM toolchain}]
# Build arm toolchain from nxtOSEK
COPY build_arm_toolchain.sh home/
RUN chmod 755 home/build_arm_toolchain.sh
RUN home/build_arm_toolchain.sh

# Add build.sh
WORKDIR /home/nxtosek
COPY build.sh ./
RUN chmod 755 ./build.sh
\end{lstlisting}

Finally, the nxtOSEK files are moved to the home folder, before the entry point for the container is set to the location in the home folder.
This means that whenever the Docker image is run, any arguments are passed along to the build file in the home directory.

\begin{lstlisting}[language=docker,label={lst:dockerimplementation4},caption={Moving the nxtOSEK files and setting the entry point}]
# Move nxtOSEK files to home folder
RUN useradd nxtosek
RUN usermod --password $(echo nxtosek openssl passwd -1 -stdin) nxtosek
RUN chown -R nxtosek ./
USER nxtosek
ENV HOME /home/nxtosek
ADD nxtOSEK.tar.xz ./
ADD ecrobot.mak nxtOSEK/ecrobot

# Set build.sh as entrypoint to execute when executing docker container
ENTRYPOINT ["/home/nxtosek/build.sh"]
CMD []
\end{lstlisting}

The Docker container allows the group to easily build the NXT implementation from any machine, as long as Docker is supported by the OS.
The Docker image is hosted publicly on the Docker Hub at \url{https://hub.docker.com/r/teknight/nxtosek/}.

In addition to local builds, the Docker image is also used on the TeamCity server, as part of the continuous integration for the project, meaning that the Docker image is used every time something is pushed to a branch.
%Maybe describe Continuous integration?

\subsection{Continous Integration}
To ensure that builds are not failing, the group set up a TeamCity server\cite{TeamCityHomepage} and combined it with GitHub Webhooks.
A GitHub webhook is sent from GitHub to the TeamCity server whenever new code has been pushed.
GitHub then provides a API endpoint for marking commits and branches as passing or failing builds, via their Status API \cite{GithubStatusAPI}.

\figur{1.0}{images/flowchart-build.png}{Our build-flow}{fig:flowchart-build}

To avoid a unmaintainable test-script, the build-process is split into sub-processes, such as the test-report script in Snippet~\ref{lst:pythonreportcheck}, that checks each line in \LaTeX{}-code for multiple sentences on a single line.
A test for this was introduced to reduce merge conflicts, as well as increasing readability.

\begin{lstlisting}[language=python,label={lst:pythonreportcheck},caption={Checking .tex files}]
#!/usr/bin/python

import os
import sys
import glob

white_list = ["eg", "i.e", "etc", "e.g"]

def get_tex_lines():
    for file in glob.iglob(os.path.dirname(os.path.realpath(__file__)) + '/report/*/**/*.tex', recursive=True):
        with open(file) as f:
            for line in f.readlines():
                yield line

def check_line(line, initial = True):
    parts = line.strip().split("(*@{.}@*) ")
    if line.startswith("%"):
        return True
    if len(parts) == 1:
        return True if (initial or "." not in parts[0]) else any(parts[0][:-1].endswith(x) for x in white_list)

    newLine = "(*@{.}@*) ".join(parts[:-1])
    for x in white_list:
        if newLine.endswith(x):
            return check_line(newLine[:-len(x)], False)
    return False

result = True
for line in get_tex_lines():
    if not check_line(line):
        print("Illegal line:", line)
        result = False

exit(0) if result else exit(1)
\end{lstlisting}

One of the advantages of using a continuous integration tool is that the merging is self-testing.
This means that when we maintain smaller code changes, that are merged into a mainline branch often, instead of monolithic merges, that have a higher likelihood of conflicting code, and long running forks.

Overall, the continuous integration has helped to ensure that unit tests were always passing and that our mainline branch is working.
This is done by running through the flowchart in Figure~\ref{fig:flowchart-build} every time some code needs to be build.
This includes building the NxtOSEK program, ensuring no compile time errors and compile the \LaTeX{} code as well.
