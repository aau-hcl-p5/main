\subsection{NXT USB implentation}
\label{sec:nxtusbimp}
In \autoref{sec:usbdes}, a high level description of the USB communication between the computer and the NXT was described.
This section will describe the specifics of the implementation of the USB receival on the NXT.

The USB receiver has 3 parts, as shown on \autoref{fig:compusb}.

\begin{lstlisting}[language=CSharp,label={lst:usbhandshake},caption={ecrobot\_device\_initialize method from nxt.c}]
    void ecrobot_device_initialize(void) {
        init_motor(NXT_PORT_A, AXIS_Y, 20);
        init_motor(NXT_PORT_B, AXIS_X, 20);
        ecrobot_init_usb();
    }
\end{lstlisting}


\autoref{lst:usbhandshake} shows the initialization of the NXT device.
The \texttt{ecrobot\_device\_initialize} function is invoked by nxtOSEK on program startup, and initializes the motors and the usb communication, with the \texttt{ecrobot\_init\_usb()} invocation.

\texttt{ecrobot\_init\_usb()} is required to use the NXT's USB port for communication.
As mentioned in \autoref{sec:usbdes}, the USB initialization posed a challenge, due to lack of documentation.

The next part of the USB receival is the actual continuous retrieval from the USB buffer.
\begin{lstlisting}[language=CSharp,label={lst:usbreceive},caption={get\_status\_code method from usb.c}]
bool get_status_code(STATUS_CODE *out_code, T_VECTOR *out_location) {
	int32_t len;
	uint8_t buffer[MAX_SIZE_OF_USB_DATA];

	/* critical section */
	GetResource(USB_Rx);
	/* read USB data */
	len = ecrobot_read_usb(buffer, 0, MAX_SIZE_OF_USB_DATA);
	ReleaseResource(USB_Rx);

	if (len == sizeof(STATUS_CODE))
	{
		memcpy(out_code, buffer, sizeof(STATUS_CODE));
		return true;
	}
	if (len == MAX_SIZE_OF_USB_DATA) {
		memcpy(out_code, buffer, sizeof(STATUS_CODE));
		memcpy(out_location, buffer + sizeof(STATUS_CODE), sizeof(T_VECTOR));
		return true;
	}
	return false;
}
\end{lstlisting}
The \texttt{get\_status\_code} function handles both the continuous read of the USB and setting the status code of the system.

The \texttt{GetResource(USB\_Rx)} call, locks the USB resource for other processes, while the \texttt{ecrobot\_read\_usb} call, reads the data from the USB into the \texttt{data} unsigned integer data.
Finally, \texttt{ReleaseResource(USB\_Rx)} releases the USB resource.


After releasing the USB resource, the length of the data is checked.

If the length the size of \texttt{STATUS\_CODE}, which is defined as an unsigned 16 bit integer in the \texttt{usb.h} header file, it is copied to the \texttt{out\_code} which is specified as an input parameter in the form of a pointer.


Alternatively, it can be of length \texttt{MAX\_SIZE\_OF\_USB\_DATA}, which is defined as a status code concatenated with a target vector, a data structure containing a set of $X$ and $Y$ coordinates in the form of two 16 bit integers.
In this case, the status code is copied to the \texttt{out\_code} and the received target information is copied to the \texttt{out\_location}, which is also specified as an input parameter.


If nothing is read from the USB, meaning that the length of the read data is $0$, it returns false, indicating that nothing was read.



The \texttt{get\_status\_code function} is called from the \texttt{receive\_data} function in the \texttt{nxt.c} file, which contains the tasks for the device.
