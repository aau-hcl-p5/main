\section{USB implentation}
\label{sec:usbimp}
In \autoref{sec:usbdes}, a high level description of the USB communication between the computer and the NXT was described.
This section will describe the specifics of the implementation of the USB receiver on the NXT.

The USB receiver has 3 parts, which is also shown in \autoref{fig:compusb}.

\begin{lstlisting}[language=CSharp,label={lst:usbhandshake},caption={ecrobot\_device\_initialize method from nxt.c}]
    void ecrobot_device_initialize(void) {
        init_motor(NXT_PORT_A, AXIS_Y, 20);
        init_motor(NXT_PORT_B, AXIS_X, 20);
        ecrobot_init_usb();
    }
\end{lstlisting}
\autoref{lst:usbhandshake} shows the initializing of the NXT device.
\texttt{ecrobot\_device\_initialize} is invoked by nxtOSEK on program startup.
The essential function call for function call is the \texttt{ecrobot\_init\_usb()} call.
This function is a part of the nxtOSEK API and is required to use the NXT's USB port for communication.
As mentioned in \autoref{sec:usbdes}, this part posed a few challenges, as it was not clearly documented.

The next part of the USB receival is the actual continuous retrieval from the USB buffer.
\begin{lstlisting}[language=CSharp,label={lst:usbreceive},caption={get\_status\_code method from usb.c}]
bool get_status_code(STATUS_CODE *out_code, T_VECTOR *out_location) {
	int32_t len;
	uint8_t buffer[MAX_SIZE_OF_USB_DATA];

	/* critical section */
	GetResource(USB_Rx);
	/* read USB data */
	len = ecrobot_read_usb(buffer, 0, MAX_SIZE_OF_USB_DATA);
	ReleaseResource(USB_Rx);

	if (len == sizeof(STATUS_CODE))
	{
		memcpy(out_code, buffer, sizeof(STATUS_CODE));
		return true;
	}
	if (len == MAX_SIZE_OF_USB_DATA) {
		memcpy(out_code, buffer, sizeof(STATUS_CODE));
		memcpy(out_location, buffer + sizeof(STATUS_CODE), sizeof(T_VECTOR));
		return true;
	}
	return false;
}
\end{lstlisting}
The \texttt{get\_status\_code} function handles both the continuous read of the USB and setting the status code of the system.

The \texttt{GetResource(USB\_Rx)} call, locks the USB resource for other processes.
It then reads the data from the USB buffer, with the \texttt{ecrobot\_read\_usb} call, into the data unsigned integer data and releases the USB resource.

After releasing the USB resource, the function checks the length of the data, which can be the length of a status code, which is defined as an unsigned 16 bit integer in the \texttt{usb.h} header file the length of \texttt{MAX\_SIZE\_OF\_USB\_DATA}.
In the case of a status code, this code is copied to the \texttt{out\_code} which is specified as an input parameter in the form of a pointer.
Alternatively, it can be of length \texttt{MAX\_SIZE\_OF\_USB\_DATA}, which is defined as a status code concatenated with a target vector, a data structure containing a set of $X$ and $Y$ coordinates in the form of two 16 bit integers.
In this case, the status code is copied to the \texttt{out\_code} and the received target information is copied to the \texttt{out\_location}, which is also specified as an input parameter.
If nothing is read from the USB, meaning that the length of the read data is $0$, it returns false to indicate that nothing was read.

The \texttt{get\_status\_code function} is called from the \texttt{receive\_data} function in the \texttt{nxt.c} file, which contains the tasks for the device.
