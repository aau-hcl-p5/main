\section{Real-time systems}
In order to guarantee that a computer systems is able function in a time critical environment i.e. car controllers and rockets its essential that the software of that computer system is true to its deadlines. 
In the following section the different methods related to real-time systems will be elaborated.

\subsection{Scheduling}
Then creating a real-time systems an essential part of making such systems is the ability to make the system schedulable.
Being scheduable means that all tasks has to be true to their deadlines and their periods. 
Multiple methods can be used to determine whether a system is schedulable or not, but the base principle is the same. 
There are tree main aspects that most general purpose scheduling algorithms emphasize:
\paragraph{Fairness} meaning that in general purpose systems it is important that all processes is granted execution time thus preventing starvation of a given process. 
\paragraph{Efficiency} meaning that since the scheduler is invoked very often in a general purpose operation system it needs to be efficient in order not to waste execution time on the CPU. 
\paragraph{Throughput} meaning the number of processes completed the fastest.
This aspect is related to efficiency hence the CPU should be on idle as little as possible. 

All these aspects is related to general purpose schedulers found in most commonly used operating systems. 
However these aspects are not relevant for real-time systems since the main focus of those systems is \textit{timing} requirements of each process.
In order to determine schedulability a real-time system it needs to have fixed set of processes which needs to be periodic in order to say that it is schedulable.


In the following some of the different methods will be elaborated. 

\subsubsection{Cyclic Executives}
One off the simplest scheduling methods is the cyclic executives method also know as timeline scheduling.
This method only allows one process or task to be executed at a time. 
Then using this method the schedule of which process is executed is determined by a \textit{major cycle} and a \textit{minor cycle}.
The \textit{major cycle} is specified by calculating the Greatest Common Divisor (GCD) between all tasks in the system.
While the \textit{minor cycle} is specified by the Least Common Multiple (LCD). 
The major cycle specifies a deadline of which all process has to be completed at least once.
The minor cycle specifies the interval of which the at least one task should have completed once. 

Since cyclic executives is a very simple scheduling method it is easy to implement and to analyse it is also one of the most commonly used in real-time systems\cite{RealTimeEmbeddedSystems}. 

\subsubsection{Task-based scheduling}
Another commonly used scheduling method in real-time systems is the task-based method. 
This method is based on tasks with priority where the task.


Then using the task-based priority scheduling method each task is given a fixed priority before run-time which can not be changed doing run-time.
The priority of a task is not an indication of the importance of the task it is a simply a priority to make the system scheduable. 
The priority of a task can be set by using a method called \textit{Earliest deadline first}.
\textbf{Earliest deadline first} means that it is the task with the earliest absolute deadline that is allowed to execute first. 
The next task to execute is the one with nearest deadline

This scheduling method can work both with and without pre-emption of tasks. 
Pre-emption allows a running task to be set on hold then a higher priority task is ready to run. 
Then using this method for scheduling a task with a low priority is allowed to run if no other task with a higher priority is ready to run. 
Given a low priority is running and a higher priority task is marked as ready, the low priority task will be pre-empted. 