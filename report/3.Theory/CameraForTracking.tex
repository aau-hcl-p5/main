\section{Using a camera for estimating direction}
As mentioned in the requirements (\autoref{subsec:requirements}), a camera will be used to locate a target.
In this section, the theory of how a camera processes a three dimensional world into a two dimensional video and how this is applicable to the project will be elaborated upon.

\subsection{The perspective of a camera}
When a camera captures a picture, it makes use of perspective projection.
This means that the camera is able to transform a world consisting of three dimensional objects into a flat picture, which can be seen on for example a computer screen.

\figur{0.7}{images/perspectiveprojection.jpg}{Idea behind perspective projection\cite{perspective}.}{fig:perspective}

\autoref{fig:perspective} shows the two objects that are captured in the view frustum and how they are represented in the viewport in a two-dimensional format.
For this project, using this format is advantageous since the primary focus is to hit a moving object with a laser.

\subsection{Hitting a target with a laser}
The primary benefit of using a laser, is that its trajectory is practically linear, and the target is hit instantly.
Rather than having to calculate the direction of the target in a three dimensional space, meaning that the distance has to be taken into account, it is instead possible to compare two images captured by the camera.
\figur{0.7}{images/estimate.jpg}{Estimation of next possible position.}{fig:estimate}

\autoref{fig:estimate} shows two subsequent frames, a and b.
The frames are used for used to find a moving object, based on the differences between the two frames.
Frame c shows that the red ball is the object in motion.
Frame d shows the next predicted location.
The predicted location is calculated based on the difference in time between frame a and frame b, and the distance the red ball has moved between the two frames.

This assumes that the red ball moves at a constant speed, which will seldom be the case, as the object is likely to be accelerating or decelerating.

As the first version of the system will use a laser as its shooting mechanism, the only information needed to hit the object is the relative distance between two frames in the image stream provided by the camera. 
When knowing the relative distance between frames, it should be a matter of moving the robot according to the relative distance.
