\section{Using a camera for estimating direction}
As mentioned in the requirements (\label{subsec:requirements}), a camera will be used to identify the presence of a moving target.
In this section, the theory of how a camera processes a three dimensional world into a two dimensional video and how this is applicable to the project will be elaborated upon.

\subsection{The perspective of a camera}
When a camera captures a picture, it makes use of perspective projection.
This means that the camera is able to transform a world consisting of three dimensional objects into a flat picture, which can be seen on for example a computer screen.

\figur{0.7}{images/perspectiveprojection.jpg}{Idea behind perspective projection\cite{persproj_pic}.}{fig:perspective}

As seen on Figure~\ref{fig:perspective} 

%skriv noget om at man kan omdanne 3d rummet til et 2d rum.
%forklar om matematikken bag perspective projection 
% https://math.stackexchange.com/questions/2305792/3d-projection-on-a-2d-plane-weak-maths-ressources

\subsection{Hitting a target with a laser}
The major benefit of working with a laser is that it is almost not affected by gravity, meaning that finding the intersection between the target and the laser is easy with the two dimensional image generated by the camera.
Rather than having to calculate the direction of the target in a three dimensional space, meaning that the distance has to be taken into account, it is simply possible to compare two images captured by the camera.

%noget om at man kan tage to billeder og finde forskelle, hvis baggrunden er statisk
%samme men med bevægende kamera, udligning af bevægelse

%

\subsection{Expanding to use a projectile}
% skriv noget om hvordan vi kan bruge projektil
% noget om at vi skal udregne distance og intersection point osv

% overgang til MI om hvordan det kan benyttes til at lave de her udreginger