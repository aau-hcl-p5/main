\section{Machine intelligence}\label{theory:machineintelligence}
Machine intelligence (MI) is the study of computational agents, acting in an intelligent manner.
The following subsections are all concerned with the principles needed to analyze and reason for an MI system.

\subsection{Modelling the domain}\label{Theory:MIModelling}
When designing a MI system, it is important to consider the domain that the MI system will affect.
An MI system is comprised of an agent, the input to the agent, and the actions it performs to affect the environment it acts in.

\figur{0.6}{images/actualagent.png}{An agent, its input and its environment{.} Source:~\cite{ArtificialIntelligencealanpoole}.}{fig:agentenvironment}

An agent, as shown in \autoref{fig:agentenvironment}, is an abstraction on an entity that acts in an environment based on its perception of said environment\cite{ArtificialIntelligencealanpoole}.
It might be intelligent in the context of acting purposefully, by trying to reach its goals by performing actions, or it might act unpurposely, where actions has no relation to goals.
The two types of agents are \textit{Purposeful Agents} and \textit{Nature}.

Examples of agents could be:
\begin{itemize}
	\item \textbf{Software agents}, which solely operate on a computational environment, i{.}e{.} it modifies a software environment.
	\item \textbf{Robotic agents}, which perceives the environment through sensors and performs actions on the environment through motors and other actuators.
	\item \textbf{Human agents}, which is quite simply a human. 
\end{itemize}

An agent chooses an action based on the input it receives.
This means that even though an agent is purposeful, it might accidentally act against its goal.
This is caused by the fact that it cannot objectively know the effect of each action, but only try to reason about them.

The different types of input are explained here:
\begin{itemize}
    \item \textbf{Abilities}, which are a definition of the capabilities of the agent, i{.}e{.} what actions the agent can perform.
    \item \textbf{Goals}, which are the preferences of the agent, meaning what goal it will attempt to fulfil.
    \item \textbf{Any prior knowledge}, which is knowledge about the agent itself and the environment it acts in.
    \item \textbf{Stimuli}, from sensors or other observations from the environment.
    \item \textbf{Past experience}, which is a summation of previous actions, data, and stimuli from the environment.
\end{itemize}


An environment is understood in the context of an agent, and can be categorized as either a physical or a computational environment.
An environment combined with an agent is called a world, and a world might contain multiple agents.
In a multi agent world, it can be beneficial for an agent to consider the inputs of other agents, which might be a complex task if the other agents act purposefully.

When designing the system, it is important to consider the world it is a part of, so that the agent can act purposefully and reach its goals.
When trying to reach the goals of the agent, different types of solutions can be found.
The \textit{optimal solution} is the best solution according to some predefined measurement of quality, meaning a solution that fully satisfies an arbitrary set of goals.
This solution is ideal, but it might not be required, and in some cases is a purely hypothetical solution, that might not be feasible in reality.
Often the \textit{satisfactory solution} is close enough to the actual solution, to be used by the agent.


An agent can be learned, or computed, through either design time computation, offline computation or online computation.
Design time computations are done by the designer of the agent, during the design of the agent.
Offline computation is performed by the agent itself, but before it has ever acted on the environment.
This mean that the agent will not learn based on its actions on the environment, but rather based on predetermined data.
Finally there is online computation, which happens between the agent's observations of the environment and the enactment of the agent's actions on the environment.
Thus, the agent will learn based on the results of its previous actions, and hopefully improve its performance.

A way to do this is through the usage of neural networks.
% måske section
The theory of those will be explored in the following subsections.
