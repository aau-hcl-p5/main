% \section{Object Recognition}
% To be able to figure out where the target is, its location within the video stream has to be determined.
% It is not enough to simply answer the binary question "is a target present?", but rather "where is the target located?", so this information can be processed and in turn move the motors to point the laser at the target.
% As mentioned in \autoref{sec:obj_tracking}, object recognition and localization algorithms can be tedious to implement by hand, and as a result of this, machine learning algorithms will be utilized.
% Although the initial target of the \texttt{F.L.A.T.} may be a simple and easily recognizable object, the intent is to create a solution that can identify more complex targets, in a potentially noisy image.
% Because of this, it is important to look into the science of object identification, which algorithms and methods exists, and what libraries are available.

% It is important to mention that all mentions of object detection in the following subsections, refer to the localization of multiple objects.

% \section{Machine intelligence}\label{theory:MI}
% A requirement of the project is to use machine intelligence as a part of the solution.
% Machine intelligence is the study of computational agents exhibiting intelligent behaviour.
% A subtopic of machine intelligence, is machine learning with is the exercise of training a computational agent to solve a task.

% The following subsection will explore neural networks, which is a way to apply machine learning

\subsection{Neural networks}
A neural network is, as the name implies, a network of neurons.
The idea is based on the neurons in the human brain, where the sum of the neurons are used to make decisions that are more complex than the decisions the individual neurons are capable of.
The primary difference is whether it is a biological brain or an artificial one. 
In computer science, a neuron can be seen as a function that processes some input and then emits the resulting value to connected neurons.

Neural networks are a powerful, but complex, method of doing machine learning.
% and simpler solutions should also be considered when exploring which method is best suited for solving the task.
One of the reasons neural networks are especially interesting in this case, lies in their ability to recognize patterns.
%Vær opmærksom på at vi måske ændrer det her til at være mere generelt MI
Pattern recognition is a fundamental part of object recognition, making neural networks an obvious option.

\subsubsection{The structure of a neural network}
A typical neural network consists of an input layer, an output layer and one or more hidden layers, between the input and output layers.
It is the task of the hidden layers to transform the input from the input layer into the desired output onto the output layer, with each layer consisting of one or more neurons.
In a normal, fully connected neural network all neurons in a layer are connected to all neurons in the previous layer.
All connections in a neural network have a weight associated with it, which acts as a simple multiplier for the output from the previous neuron. 
Apart from the weights, all neurons, except for input neurons, are assigned a bias, which is added to the sum of all inputs to the neuron. 
The bias is sometimes thought of as a threshold, indicating if the neuron should fire or not.

A simple example of a neural network can be seen in \autoref{fig:nnimage}.

\figur{0.7}{images/NN.png}{Illustration of a simple neural network with 2 neurons in each layer\cite{NeuralNetworkImage}.}{fig:nnimage}

Before passing the result of a neuron on, the result is first transformed by an activation function. 
The activation function is required in order for the neural network to learn to solve a complex non-linear model. 
This will be further discussed in \autoref{training-a-neural-network}, which discusses the training of a neural network.

The following paragraphs explains a few common activation functions.
\paragraph{Identity}
Identity is the simplest activation function, it is an activation function that does not alter the input. 
This is also known as an identity function and can be used when learning simple models, but fails to learn complex non-linear models.

\paragraph{Perceptron}
A perceptron is a neuron that has an activation function that returns a binary signal, either zero or one, depending on the input value. 
Usually any positive value will be a one, and all negative values will be a zero.

\paragraph{Logistic sigmoid}
The sigmoid function is a commonly used activation function in neural networks.
It is defined as:

$$f(x) = \frac{1}{1 + e^{-x}}$$

It maps the input value to a value between one and zero, and is especially useful for learning, due to its non-linear slope.
A visual representation can be seen in \autoref{fig:sigmoid}.


\definecolor{ffqqqq}{rgb}{1,0,0}

\tikzimage{[line cap=round,line join=round,>=triangle 45,x=1.0cm,y=2.5cm]
	\draw[->,color=black] (-5,0) -- (5,0);
	\foreach \x in {-4,-2,2,4}
	\draw[shift={(\x,0)},color=black] (0pt,2pt) -- (0pt,-2pt) node[below] {\footnotesize $\x$};
	\draw[->,color=black] (0,-3) -- (0,3);
	\foreach \y in {-3,-2,-1,1,2}
	\draw[shift={(0,\y)},color=black] (2pt,0pt) -- (-2pt,0pt) node[left] {\footnotesize $\y$};
	\draw[color=black] (0pt,-10pt) node[right] {\footnotesize $0$};
	\clip(-5,-3) rectangle (5,3);
	\draw[line width=2pt,color=ffqqqq, smooth,samples=100,domain=-5.0:5.0] plot(\x,{1/(1+2.718281828^(-(\x)))});}{Logistic sigmoid function plotted from -5 to 5}{fig:sigmoid}


\paragraph{RELU}
Rectified linear unit is a modern activation function, that has proven to be powerful in deep learning networks. 

It is defined as:

$$f(x) = max(0, x)$$

A visual representation can be seen in \autoref{fig:relu}.

\tikzimage{line cap=round,line join=round,>=triangle 45,x=1cm,y=1.3cm]
		\draw[->,color=black] (-10,0) -- (10,0);
		\foreach \x in {-10,-8,-6,-4,-2,2,4,6,8}
		\draw[shift={(\x,0)},color=black] (0pt,2pt) -- (0pt,-2pt) node[below] {\footnotesize $\x$};
		\draw[->,color=black] (0,-3) -- (0,10);
		\foreach \y in {-2,2,4,6,8}
		\draw[shift={(0,\y)},color=black] (2pt,0pt) -- (-2pt,0pt) node[left] {\footnotesize $\y$};
		\draw[color=black] (0pt,-10pt) node[right] {\footnotesize $0$};
		\clip(-10,-3) rectangle (10,10);
		\draw [line width=2pt,color=ffqqqq] (-10,0)-- (0,0);
		\draw [line width=2pt,color=ffqqqq] (0,0)-- (10,10);}{RELU plotted from -10 to 10}{fig:relu}


\subsubsection{Calculations in a neural network}\label{sec:calculationsNN}

The output of a neuron is calculated by summing the output of all the connected neurons multiplied with the associated weight plus the bias and then passed through the activation function.
In \autoref{fig:nntrain} the value of $h1 $ is calculated as:
$$\phi(i_1 \cdot w_1 + i_2 \cdot w_3 + b_1)$$

\figur{0.7}{images/NNTrain.png}{Calculating value of h1 \cite{NeuralNetworkImage}.}{fig:nntrain}

More formally the outputs of a layer can be calculated by matrix multiplication:

$$
\phi\left(
\begin{bmatrix}
w_{0,0} & w_{0,1} & \dots  & w_{0,n} \\
w_{1,0} & w_{1,1} & \dots  & w_{1,n} \\
\vdots & \vdots & \ddots & \vdots \\
w_{k,0} & w_{k,1} & \dots  & w_{k,n}
\end{bmatrix}
\cdot
\begin{bmatrix}
o^l_0\\
o^l_1\\
\vdots \\
o^l_n
\end{bmatrix}
+
\begin{bmatrix}
i^l_0\\
i^l_1\\
\vdots \\
i^l_n
\end{bmatrix}
\right)
$$

This makes it easier to reason about the calculations of the network as a whole.

\subsubsection{Training a neural network}\label{training-a-neural-network}
When doing machine learning with a neural network, all the weights and biases are usually initialized with random values, which will result in the initial iterations returning bad results.

This is solved by letting the neural network train itself, in order to learn it how to complete the desired task.

The neural network should try to minimize the amount of errors on the output layer.
The error is calculated using a loss function.
The loss function can be defined in multiple ways, but a simple loss function squares the difference between the actual output and the expected output.

\paragraph{Back propagation}
Back propagation is a common way of training a neural network.
The error from the output layer is back propagated through the network.
In order to minimize the error at the output layer, all weights and biases are adjusted throughout the network.
Back propagation is often used along with gradient descent, in order to determine how much each weight and bias should be updated.

\paragraph{Gradient Descent}
With gradient descent (GD) one can use the derivative of a function to minimize the error.

To calculate how much a weight $w$, or bias for that matter, affects the total error of the network one should find the derivative of the function with respect to $w$.
This means that $w$ will affect the total error by:

$$\frac{\partial Error}{\partial w}$$

This has to be done for all weights and biases in the network.
Once this derivative is found, weights and biases are updated.
The value with which the weight or bias should be updated equals the negative value of the derivative multiplied by a learning rate $\alpha$.
This will reduce the amount with which the weight or bias affects the total error, since it descents towards a minimum.

Unfortunately, it is not guaranteed to move towards the global minimum, but rather a local minimum.
This is illustrated in \autoref{fig:sgd} where $A$ and $B$ will descent towards the local minimum.

\definecolor{xdxdff}{rgb}{0.49,0.49,1}
\figur{0.8}{images/sgd.png}{Gradient descent illustration}{fig:sgd}

Although this may not render the best results, the results gained by using gradient has proven to be good enough to be useful.

\subsection{Convolutional Neural Network}
\label{sec:cnn}
Convolutional Neural Networks (CNN) are often used for image classification, and is a popular type of neural network in regards to object detection and image processing.
In challenges such as the Imagenet Large Scale Visual Recognition Challenge (LSVCR), the majority of the entries are done using CNN \cite{ILSVRC_Results}. 

The special thing about a CNN is the addition of convolutional layers.
The convolutional layers will add convolutional filters, that has the main purpose of acting as a kind of feature detector.
The filters works by combining $n \times n$ inputs into a single output.
The output can resemble some feature found within the convolutional layer.
\autoref{fig:cnn} and \autoref{fig:cnn_feature} illustrates the concepts from CNN.

\figur{0.8}{images/cnn}{Convolution layer illustration\cite{cnnguide}}{fig:cnn}

\figur{0.8}{images/cnn-feature}{Feature identification\cite{cnnguide}}{fig:cnn_feature}

The next layers in the neural network can then use these feature outputs in order to better reason about the image.

An example of object recognition could be facial recognition.
The first layer might recognize lines, edges or similar low level features.
The following layers might detect increasingly complicated facial features, ending at some layer, in which one neuron would transmit a signal if a nose is present in the image, and another neuron transmits if a mouth is present.

\figur{0.6}{images/cnn_face}{A simplified example of the abstractions within a CNN\cite{CNNStructure}.}{fig:face_cnn}

However, it is important to understand, that this is not necessarily the patterns that a CNN will detect within its layers.
Actually it is far more likely, that the layers will recognize something that to a human has no relation to a face, but the computer will have found a correlation between these seemingly random patterns and a face.

CNNs are often used along with pooling.
Pooling is a concept where multiple outputs are combined into a single neuron in the next layer.
This can be done in order to down-sample the input and reduce the complexity, making the CNN run faster.
Some common examples of pooling is max pooling and average pooling, in which the average or max value is output from a number of input neurons.

CNNs are an effective choice for classification, meaning that it can answer whether the image is of a car.
Now the question is whether it can be used to figure out the position of the objects within the image.
The answer can be found by looking into Regional Convolutional Neural Network.


\subsection{Regional Convolutional Neural Network}
A Regional Convolutional Neural Network (RCNN) utilizes CNN to get both a classification along with a bounding box, which in turn can be used to determine the center of the location.
It achieves this by applying the CNN to smaller regions of the image. 
These regions are extracted from the image using a method called selective search.
If the region contains an element recognized by the CNN, a bounding box can be drawn around the region.
The bounding box can finally be tightened using different methods, a popular option being a linear regression model\cite{CNNHistory}.

Unfortunately RCNN in itself is not viable for real time object detection.
However, there have been a lot of modifications and improvements of RCNN that improves the performance of it.
An improved version is called \textit{faster RCNN}, which is leaning towards being feasible for real time object detection, as it improves the performance of RCNN considerably\cite{fasterRCNN}.

\subsection{You Only Look Once}\label{sec:YOLOAnalysis}
An alternative to RCNN is You Only Look Once (YOLO) which is a new technology developed with real-time processing in mind.
Benchmarks shows, that YOLO can do real-time object detection using modern high-end consumer GPU's.
Furthermore the performance of YOLO beats that of \textit{faster RCNN}\cite{odDetection}.

YOLO also uses CNN like RCNN, but takes a fundamentally different approach to object detection.
YOLO will partition the image into $S\times S$ tiles, where each tile should determine $N$ bounding boxes, along with a confidence level for the given bounding box, signifying the confidence that a given bounding box actually contains an object.
YOLO will use a CNN in order to obtain a classification score for each bounding box.
The classifications scores along with the bounding box confidence is used to calculate the final probability score of a class being present in a given bounding box\cite{odDetection}.

As a result of this a total of $S\times S\times N$ tiles are given a probability score.
In order to only see the most relevant ones, a threshold should be set, so only the tiles with the highest probability is shown.

The process is visualized in \autoref{fig:yolo}.

\figur{0.9}{images/yolo.jpg}{A visual representation of YOLO\cite{odDetection}.}{fig:yolo}

\subsection{GOTURN}
% https://www.learnopencv.com/goturn-deep-learning-based-object-tracking/
Generic Object Tracking Using Regression Networks (GOTURN), is an algorithm based on deep learning\cite{goturn}.

GOTURN learns about motion tracking and object identification patterns, by utilizing a large set of videos of different objects moving.
The model is then shipped with the algorithm implementation, and therefore there is no need for online learning, making it more efficient.


The benefit of this, is that it runs faster, as it does not have to ever modify the model or analyze whether the given result is relevant.
However, this comes at a cost. 


The algorithm has a tendency to favor objects within the training set, even if they remain stationary.
For instance, if tracking a bird flying past stationary planes, it might lock onto the planes rather than the bird, even though the planes are not moving, if the planes were in the training set.
This is due to the unpredictability of what the algorithm see as a correlation between video and a correct answer.


The GOTURN algorithm utilizes CNN, which is explained in \autoref{sec:cnn}.
\figur{0.9}{images/GOTURN-architecture.jpg}{How the algorithm utilizes CNN\cite{goturn}.}{fig:goturn-arch}
GOTURN works by looking at two frames; the first, also known as the previous frame, and the second, also known as the current frame.
This is also shown on \autoref{fig:goturn-arch}.
The premise is that the location of the object is known in the previous frame, and a bounding box is present.
Based on the bounding box of the previous frame, the current frame is cropped to two times the size of the previous bounding box.
A CNN is then trained to predict the bounding box in the second frame.

Possible problems with this algorithm is that the location of the target is unknown initially, as the target will not necessarily be in the picture at startup time, and will eventually leave the image feed again, as the \texttt{F.L.A.T} will have idle time.
Another possible issue would be depending on the speed of the given target and the frame-rate of the camera, as the object might have left the cropped area. 
This could be solved by a larger crop, however this makes it less relevant.

\subsection{TensorFlow}
TensorFlow is a framework for working with machine learning\cite{tensorflow2015-whitepaper}.
The framework was initially developed for Python, but has recently implemented support for other languages.
It is developed by the Google Brain team, and owned by Google Inc.\cite{tensorflow-attribution}.
It has a high-level API, called Keras, and a low-level API, which makes it possible to use as an entry-level user, making it ideal for hobby and educational purposes, but it is also usable for complex production-level applications.

TensorFlow uses the concept of tensors to abstract the basis of computation, and makes it possible to write machine learning and machine intelligence with fewer steps.

A tensor is a mathematical entity, and the mathematical definitions are outside the scope of this project.
In the context of TensorFlow, it is enough to think of it as an entity in a mathematical process.
This might be a constant primitive value, or a mapping from one or more primitive values to one or more primitive values.

To better show the example of tensors and how TensorFlow operates, the low-level example in \autoref{lis:TensorFlowPy1} is used in the TensorFlow tutorial\cite{tensorflow2015-whitepaper}.

\begin{lstlisting}[language=Python,label=lis:TensorFlowPy1,caption=Example of a TensorFlow program]
a = tf.constant(3.0)
b = tf.constant(4.0)
total = a + b
print(a)
print(b)
print(total)
\end{lstlisting}
The reader might assume that the output would be 3,4, and 7, respectively, however the output is as seen in \autoref{lis:TensorFlowPy2}:
\begin{lstlisting}[language=Python,label=lis:TensorFlowPy2,caption=Output of example \autoref{lis:TensorFlowPy2}]
Tensor("Const:0", shape=(), dtype=float32)
Tensor("Const_1:0", shape=(), dtype=float32)
Tensor("add:0", shape=(), dtype=float32)
\end{lstlisting}
This is because the $ a + b $ outputs another tensor, which can be evaluated outputting the result 7.
This is is because $a$ or $b$ might not be a constant, but a placeholder, which would be supplied at run-time, so building these tensors creates a model that can then be supplied data and then evaluated.

Tensors can also be tuples of different shapes, and when applying levels of abstraction, which is what Keras does, it is possible to use it for image recognition and other more complex problems.

\subsection{OpenCV}
OpenCV(Open Source Computer Vision Library), is a free-to-use library for C++, Python, and Java, which can be used for a lot of common computer vision tasks\cite{opencv}.

From OpenCV's about page:
\begin{quote}
\textit{OpenCV (Open Source Computer Vision Library) is an open source computer vision and machine learning software library.}
\textit{OpenCV was built to provide a common infrastructure for computer vision applications and to accelerate the use of machine perception in the commercial products\cite{opencvabout}.}
\end{quote}
According the above quote OpenCV should be easy to work with, and it should be easy and quick to implement different image manipulation tasks programmatically.
Examples of tasks include\cite{opencvexamples}:
\begin{itemize}[noitemsep]
	\item Individual pixel manipulation, for example accessing all pixels of a certain color and changing them to another color, e.g. basic read/write.
	\item Different forms of image processing, such as rotating the image, smoothing the image color space, image thresholding and more.
	\item Feature and object detection, a more advanced image analysis that can find objects and features in image. 
	Examples of usage here could be facial detection or detecting various objects in the image.
\end{itemize}

The below snippet is an example of facial recognition, implemented using OpenCV.
\begin{lstlisting}[language=Python,label=lis:opencvExample,caption=Source code from the OpenCV documentation\cite{opencvHaarExample}.]
	import numpy as np
	import cv2
	
	face_cascade = cv2.CascadeClassifier('haarcascade_frontalface_default.xml')
	eye_cascade = cv2.CascadeClassifier('haarcascade_eye.xml')
	
	img = cv2.imread('sachin.jpg')
	gray = cv2.cvtColor(img, cv2.COLOR_BGR2GRAY)
	
	faces = face_cascade.detectMultiScale(gray, 1.3, 5)
	for (x, y, w, h) in faces:
	  cv2.rectangle(img, (x, y), (x + w, y + h), (255, 0, 0), 2)
	roi_gray = gray[y: y + h, x: x + w]
	roi_color = img[y: y + h, x: x + w]
	eyes = eye_cascade.detectMultiScale(roi_gray)
	for (ex, ey, ew, eh) in eyes:
	  cv2.rectangle(roi_color, (ex, ey), (ex + ew, ey + eh), (0, 255, 0), 2)
	
	cv2.imshow('img', img)
	cv2.waitKey(0)
	cv2.destroyAllWindows()
\end{lstlisting}
\autoref{lis:opencvExample} shows an example of simple facial recognition on an image, using Haar-Cascade.
Haar-Cascade is a classifier, which has been trained to detect a specific object, in this case, faces.

While OpenCV is not a machine learning library, it does include different machine learning elements, for example k-Nearest Neighbor (kNN) classification and k-Means clustering, based on images.
kNN is a simple classification algorithm, that attempts to group different data into different classes, by looking at the data that most closely resembles each other.
K-Means is another classification algorithm, that instead attempts to find a ''center'', a mean, and then assigning all the data that lies closest to this ''center'' to that class.

The two algorithms share the trait, that they are often performed iteratively, meaning the same dataset are iterated over several times, leading to a more accurate classification, often by moving the classification of a single data-element.
This shifts the ''weight'' of the class.

Below is an example of machine learning using the OpenCV library:

\begin{lstlisting}[language=Python,label=lis:opencvMLExample,caption=Source code OpenCV documentation\cite{opencvMLexamples}.]
	import cv2
	import numpy as np
	import matplotlib.pyplot as plt

	# Load the data, converters convert the letter to a number
	data = np.loadtxt('letter-recognition.data', dtype = 'float32', delimiter = ',',
	  converters = {
	    0: lambda ch: ord(ch) - ord('A')
	  })

	# split the data to two, 10000 each
	for train and test
	train, test = np.vsplit(data, 2)

	# split trainData and testData to features and responses
	responses, trainData = np.hsplit(train, [1])
	labels, testData = np.hsplit(test, [1])

	# Initiate the kNN, classify, measure accuracy.
	knn = cv2.KNearest()
	knn.train(trainData, responses)
	ret, result, neighbours, dist = knn.find_nearest(testData, k = 5)

	correct = np.count_nonzero(result == labels)
	accuracy = correct * 100.0 / 10000
	print(accuracy)
\end{lstlisting}

\autoref{lis:opencvMLExample} shows a working ML agent that detects different letters, based on a feature data-set.
The program itself is quite simple, and the complex aspects of the program are handled by the OpenCV library.

As can be seen from both examples, it is easy to do complex image manipulation in a simple manner.

These two examples are just a small sample of the capabilities of OpenCV, and even without the machine learning aspect, OpenCV also provides excellent webcam interfacing, and image manipulation utilities.
Thus, OpenCV is an excellent solution, that could potentially be used to solve the object detection aspect of this project.