\section{Location prediction of a moving object}
To predict the location of a moving object, an observation that has to be made is whether the object is following a linear trajectory or a ballistic trajectory.

\subsection{Linear trajectory}
Given that the target follows a linear trajectory, calculating the next locations is simple.  
When an object is traveling with constant speed, the distance traveled by the object can be described as follows:
\begin{equation}\label{eqn:Distance}
d = s \cdot t
\end{equation}
$ d $ is the distance traveled, $ s $ is the speed and $ t $ is the time passed.
This means that if an object has traveled at $ 1 m/s $ for $ 1 d $ it will have traveled 1 meter. 

Based on \autoref{eqn:Distance} the variable $ s $ can be isolated resulting in:
\begin{equation}\label{eqn:Speed}
s = \cfrac{d}{t}
\end{equation}
With \autoref{eqn:Speed} the speed of the target can be calculated. 
In order to calculate the future position of the target, its previous location is needed, which will be solved using image recognition.
