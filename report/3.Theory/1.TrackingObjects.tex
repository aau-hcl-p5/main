\section{Location prediction of a moving object}
In order to predict the location of a moving object, an observation that has to be made is whether the object is flying, meaning that the object is following a ballistic trajectory or if the object is driving i.e{.} following a linear trajectory. 
In the following section elaborations will be made on these distinctions, in order to gain a deeper understanding on the subject and being able to complete the previously specified objectives for the project. 
\subsection{Linear trajectory}
Given that the target follows a linear trajectory, a simple method to calculate the new location of the target can be applied.  
In order to predict the targets future location, an understanding of basic physics of kinematics is required. 
That is, given that an object is traveling with constant speed, the distance traveled by the object can be described as follows:
\begin{equation}\label{eqn:Distance}
s = v \cdot t
\end{equation}
where $ s $ is the distance traveled, $ v $ is the velocity and $ t $ is the time passed.
Using this formula, it is easily calculated that if an object has traveled $ 1 m/s $ for $ 1 s $ will have traveled 1 meter. 
Based on Equation~\ref{eqn:Distance} the variable $ v $ can be isolated resulting in:
\begin{equation}\label{eqn:Speed}
v = \cfrac{s}{t}
\end{equation}
By applying Equation~\ref{eqn:Speed} the speed of the target can be calculated. 
In order to calculate the future position of the target, its previous location is needed, which will be solved using image recognition.

%\subsection{Ballistic trajectory}
%In order to not just track a flying object in front of the \texttt{F.L.A.T}, it is necessary to gain a basic understanding of ballistic trajectories in order to predict where the object will be after a given period of time. 
%
