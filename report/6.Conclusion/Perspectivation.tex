\section{Discussion and perspectivation}
To conclude upon the success of the project, the MoSCoW analysis and the test cases formulated in Subsection~\ref{subsec:requirements} will be used.

\subsection{MoSCoW conclusion}
In the MoSCoW analysis, a set of prioritized requirements were formulated.
In order for the project to be deemed successful, all of the \textit{must have} requirements must be fulfilled and most of the \textit{should have} requirements as well.
%MOSCOW

\subsubsection{Must have}
The must have requirements were as follows:
\begin{itemize}
	\item Ability to recognize and track a moving target in front of the device.
	\item Ability to follow a large target being dropped from 2 meters height at a distance of 3 to 4 meters away from the device, with a laser.
	\item Move fast enough to keep the laser on a moving target in real time.
	\item Ability to move the laser at least 60 degrees in both horizontal and vertical directions from the origin.
	\item Enough structural stability to be able to move without breaking.
\end{itemize}

The robot is able to recognize and track a red balloon being dropped from a height of 2 meters at distance of 3 meters from the device in real-time, with the laser being toggled off as soon as the turret no longer aims at the balloon.
The robot is able to move approximately 270 degrees on the horizontal axis, allowing a movement of about 135 degrees in each direction from the point of origin.
On the vertical axis, it is able to move a total of 140 degrees, allowing 70 degrees movement in each vertical direction from the point of origin.
Finally, the robot is able to do the above without causing instability or breakage in the model, resulting in all five \textit{must have} being accomplished.


\subsubsection{Should have}
The should have requirements were as follows:
\begin{itemize}
	\item Ability to hit a larger moving target in the air at a low speed at a distance of 3 meters.
	\item Ability to hit a small moving target in the air at a low speed at a distance of 1 to 3 meters.
	\item Ability to predict the next location of the object moving along a non-linear path, e{.}g{.} a target that continuously moves back and forth on the vertical axis while moving linearly on the horizontal axis.
	\item Ability to predict the next location of the object moving solely along the horizontal axis with a constant speed.
\end{itemize}

As mentioned in the must have conclusion, the robot is able to track a balloon falling at a distance of 3 meters.
The robot is also able to track a large object, such as a balloon, being moved at a slow speed with unpredictable movement patterns.
At a distance of 1 meters, an object with a size of at least 15 centimeters in diameter can be tracked with a good amount of precision.
Tracking smaller objects would require more precise motors as the imprecision of the motors causes some issues, as they can only move a full degree per step, meaning the movement of the laser will become more imprecise as the distance to the target increases.
For example, at a distance of 30 centimeters from the laser, one degree movement will result in the laser moving 0.3 centimeters, however, at a distance of 2.5 meters, moving a single degree will result in the laser moving more than 4 centimeters.
This limitation means that at a distance, hitting a small object is simply not viable due to the limitations of the motors.
Likewise, the ability to predict the next location of an object was relinquished as it was not deemed necessary when not firing a projectile.
Thus, 2 of the 4 \textit{should have} requirements have been fulfilled.

\subsubsection{Could have and won't have}
From the \textit{could have} section, the following requirements are formulated:
\begin{itemize}
	\item Ability to calculate the distance to the target object.
	\item Calculate the angle and power necessary to hit the target with a projectile.
\end{itemize}

And in the \textit{won't have}:
\begin{itemize}
	\item Ability to actually take down targets.
	\item Ability to differentiate between the type of object that is being observed.
\end{itemize}

None of these requirements have been fulfilled, as they are largely related to firing a projectile.

