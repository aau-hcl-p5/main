\section{Discussion and perspectivation}
To conclude upon the success of the project, the MoSCoW analysis and the test cases formulated in Subsection~\ref{subsec:requirements} will be used.

\subsection{MoSCoW conclusion}
In the MoSCoW analysis, a set of prioritized requirements were formulated.
In order for the project to be deemed successful, all of the \textit{must have} requirements must be fulfilled and most of the \textit{should have} requirements as well.
%MOSCOW

\subsubsection{Must have}
The must have requirements were as follows:
\begin{itemize}
	\item Ability to recognize and track a moving target in front of the device.
	\item Ability to follow a large target being dropped from 2 meters height at a distance of 3 to 4 meters away from the device, with a laser.
	\item Move fast enough to keep the laser on a moving target in real time.
	\item Ability to move the laser at least 60 degrees in both horizontal and vertical directions from the origin.
	\item Enough structural stability to be able to move without breaking.
\end{itemize}

The robot is able to recognize and track a red balloon being dropped from a height of 2 meters at distance of 3 meters from the device in real-time, with the laser being toggled off as soon as the balloon leaves its field of view. 
The robot is able to move approximately 270 degrees on the horizontal axis, allowing a movement of about 135 degrees in each direction from the point of origin.
On the vertical axis, it is able to move a total of 140 degrees, allowing 70 degrees movement in each vertical direction from the point of origin.
Finally, the robot is able to do the above without causing instability or breakage in the model, resulting in all five \textit{must have} being accomplished.


\subsubsection{Should have}
The should have requirements were as follows:
\begin{itemize}
	\item Ability to hit a larger moving target in the air at a low speed at a distance of 3 meters.
	\item Ability to hit a small moving target in the air at a low speed at a distance of 1 to 3 meters.
	\item Ability to predict the next location of the object moving along a non-linear path, eg{.} a target that continuously moves back and forth on the vertical axis while moving linearly on the horizontal axis.
	\item Ability to predict the next location of the object moving solely along the horizontal axis with a constant speed.
\end{itemize}

As mentioned in the must have conclusion, the robot is able to track a balloon falling at a distance of 3 meters.
The robot is also able to track a large object, such as a balloon, being moved at a slow speed with unpredictable movement patterns.
At a distance of 1 meters, an object with a size of at least 15 centimeters in diameter can be tracked with a good amount of precision.
Tracking smaller objects would require more precise motors as the imprecision of the motors causes some issues, as they can only move a full degree per step, meaning the movement of the laser will become more imprecise as the distance to the target increases.
For example, at a distance of 30 centimeters from the laser, one degree movement will result in the laser moving 0.3 centimeters, however, at a distance of 2.5 meters, moving a single degree will result in the laser moving more than 4 centimeters.
This limitation means that at a distance, hitting a small object is simply not viable due to the limitations of the motors.
Likewise, the ability to predict the next location of an object was relinquished as it was not deemed necessary when not firing a projectile.
Thus, 2 of the 4 \textit{should have} requirements have been fulfilled.

\subsubsection{Could have and won't have}
From the \textit{could have} section, the following requirements are formulated:
\begin{itemize}
	\item Ability to calculate the distance to the target object.
	\item Calculate the angle and power necessary to hit the target with a projectile.
\end{itemize}

And in the \textit{won't have}:
\begin{itemize}
	\item Ability to actually take down targets.
	\item Ability to differentiate between the type of object that is being observed.
\end{itemize}

None of these requirements have been fulfilled, as they are largely related to firing a projectile.

\subsubsection{Conclusion}
In conclusion, all of the must have requirements and half of the should have requirements have been fulfilled.
This is enough to deem the project successful, as most of the unfulfilled requirements are related to firing a projectile, which was excluded from the scope of the project. 

\subsection{Test cases}
Along with the MoSCoW analysis, four test cases were presented: 
\begin{itemize}
    \item Dropping a balloon.
    \item Rolling a ball.
    \item Throwing a ball.
    \item Shooting a balloon.
\end{itemize}

The first case, dropping a ball, was already presented in the MoSCoW conclusion and deemed successful.
The test was conducted with a heart shaped, red balloon with a diameter of 30 centimeters, which was dropped at a distance of 4 meters and a height of 2 meters. 
\texttt{F.L.A.T.} was able to keep it's laser on the target most of the time during the fall. 
For the rolling a ball test case, \texttt{F.L.A.T.} was able to roughly keep it's laser on a ball with a diameter of 52 mm successfully. 
However, the final two cases with a throwing a ball and shooting a balloon were not accomplished, as the device was not fast enough to process the speed of the object, and does not have the capabilities to shoot a balloon.

\subsection{Machine intelligence}
Several observations were made during the development of the MI aspect of the solution.
Initially, the idea for the project was to recognize objects in real-time using machine learning, however as described in \autoref{des:sec:performance}, the object recognition algorithms available, were not performant enough given the hardware limitations.
Instead, the variable motor power problem, described in \autoref{subsec:variable_motor_power}, turned out to be an excellent way to utilize MI for the solution.
The calibration module was implemented using a neural network, which was executed on the NXT.
This proved to be an excellent solution to the variable motor power problem, and a great way to use MI in the solution.

An important lesson learned from this project was that the usage of MI required thought about the domain in which the MI is applied to.


\subsection{Real-time requirements}
For the real-time part of the system, several observations were made during the development of the project.
First of all, the amount of tasks that were running on the NXT in order to make it functional were far from enough to give a good CPU utilization using the cyclic executive scheduling model.
In order to optimize the utilization time, parts of the machine intelligence calculations responsible for calculating the amount of power required to move were moved to the NXT for execution.
This resulted in the cyclic executive model being unfeasible, as there was a task that took more than 1 millisecond to run while another task had a deadline of 1 millisecond, thus making it unschedulable. 
To fix this, the scheduling model was changed to fixed priority scheduling with preemption.
The optimal choice for a system like this would be the earliest deadline first scheduling model, but this is not implementable in the nxtOSEK operating system, as this is a fixed priority operating system.

In a real world implementation of \texttt{F.L.A.T.} there are two alternative approaches that makes more sense in terms of CPU utilization: \\
The most obvious solution would be to use a device with a less powerful CPU to avoid that the CPU is idle, or alternatively use a much more powerful device, allowing the whole machine intelligence part of the system to be run on the same device, instead of offloading it to another machine.

In addition to the problems with the CPU utilization, a real world implementation should make use of better components to allow for better precision.

\subsection{Summary}
The primary problems that were experienced during the project period were attributed to the LEGO NXT platform and the nxtOSEK operating system.
Likewise, the initial thoughts for the machine intelligence were deemed infeasible due to the processing power required to perform object recognition and classification. 
In conclusion, the project should be possible to apply to a real world situation given the right hardware.
