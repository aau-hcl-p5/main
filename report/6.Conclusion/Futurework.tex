\section{Future work}\label{sec:futurework}
Throughout the development of a project, there are often several ideas that are deprioritized for one reason or another.
This section will describe the ideas that the group had, including initial thoughts and how they might have been implemented or why they were not implemented.


\subsection{Predict target location in x frames}
A \textbf{Should have} requirement from the MoSCoW analysis in Section~\ref{subsec:requirements}, was the ability to predict the location of the object in upcoming frames.
In the actual implementation this was not done, due to time constraints.
However, to hit a projectile, this would be a requirement.
The initial idea was to train the model by simply moving the object in front of a camera and tracking how the object moved.
This could lead to an additional problem, based on different movement patterns of different object types.
For this, classification could be a solution, as different objects have different movement patterns, unless the system is still based on simple inanimate objects.

These aspects could be an interesting way to explore online learning and clustering so movement patterns, for instance a ball bouncing against the ground, could be learned in real time. 
In this context, calibration could potentially also be done online, however, depending on the physical design, calibration might not be relevant, from a solution aspect.



\subsection{Firing a projectile}
During the MoSCoW analysis, the ability to hit a moving target with a projectile was classified as a \textbf{Could have} feature.
The ability to do this requires further development of the physical aspects of the solution.
Creating a device that can fire a projectile, and ideally not require a manual reload between each shot, requires a larger construction, which would put more weight on each motor leading to reduced precision.
Furthermore, the solution would require a prediction algorithm.
Prediction where to shoot would rely on the following data, as explained in section~:
\begin{enumerate}
	\item Speed of the projectile
	\item Distance to the target
	\item Speed of the projectile
\end{enumerate}
This would, therefore, require more sensors, which would add levels of difficulty to the physical domain.
This would also require more data analysis, making the calculations more abstract, than the current model of a two-axis system.


\subsection{Machine Learning instead of color detection}

Currently, the device uses OpenCV color detection, as described in Section~\ref{solution:objfillalgo}.
There were a multitude of reasons for this, the primary one being that the hardware available, a Raspberry Pi, was simply not fast enough, as described in Section~\ref{des:sec:performance}.
However, the current solution has a couple of problems.

The original reasoning behind the project was to prevent bird strikes at airports, wherein the ability target the bird but not a plane, would be an essential requirement, when implementing the device for real world use.
Another issue is that the current solution would target a person wearing a red shirt, which could potentially be hazardous.

This could be solved by using a machine learning solution, potentially the on used in Section~\ref{des:sec:performance}, however this would require a more powerful hardware.
A different solution could be the implementation of a custom neural network that is focused on the specific type of target, unlike YOLO which has multiple classification types.

Working with images on computers is typically best done on the GPU as it is actually designed for the purpose, a requirement for the real world device would be a powerful GPU.
YOLO recommends a GPU in the range of an Nvidia Titan, but a less powerful GPU could potentially do the job.

\subsection{Physical Design}
Currently, the structure is build with LEGO, which has some limitations leading to physical instability of the turret.
One of the largest sources of lacking stability are the axes.
The LEGO gears has some spacing between each element, and introducing ball bearings might improve the quality, while also creating a sturdy structure near the moveable parts.
A general stability could be introduced by creating a custom structure out of a moulded structure.
Using more powerful motors, and motors with a higher rate of precision, both regarding power and step-size, as we are currently limited by how far away a target can be, by the inability to detect movement less than 1 degree.

\subsection{Hardware Choice}
The solution could potentially utilize a more powerful device, that could handle both image processing as well as movements, making the device more simple in nature, and also more interesting in the context of real-time systems.
However, doing the current calculations on the NXT does not appear feasible, as the calculations could not be done fast enough on the Raspberry Pi, as shown in Section~\ref{des:sec:performance}.
 
\subsection{Machine learning for all movement}
As of right now, machine learning is only utilized to find a minimum power required to move.
However, a potential improvement would be to have the robot learn all the movement by itself.
Given the current motor revolutions and the distance to the target, the robot should learn itself how much power to apply to the motors.
This could potentially be achieved using some reinforced learning, where the reward would equal the delta distance to the target after moving.
