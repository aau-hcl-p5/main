\section{Conclusion}\label{Conclusion:Conclusion}
In conclusion, all of the must have requirements and half of the should have requirements have been fulfilled.
This is enough to deem the project successful, as most of the unfulfilled requirements are related to firing a projectile, which was excluded from the scope of the project. 

\subsection{Test cases}
Along with the MoSCoW analysis, four test cases were presented: 
\begin{itemize}
    \item Dropping a balloon.
    \item Rolling a ball.
    \item Throwing a ball.
    \item Shooting a balloon.
\end{itemize}

The first case, dropping a ball, was already presented in the MoSCoW conclusion and deemed successful.
The test was conducted with a heart shaped, red balloon with a diameter of 30 centimeters, which was dropped at a distance of 4 meters and a height of 2 meters. 
\texttt{F.L.A.T.} was able to keep it's laser on the target most of the time during the fall. 
For the rolling a ball test case, \texttt{F.L.A.T.} was able to roughly keep it's laser on a ball with a diameter of 52 mm successfully. 
However, the final two cases with a throwing a ball and shooting a balloon were not accomplished, as the device was not fast enough to process the speed of the object, and does not have the capabilities to shoot a balloon.

\subsection{Machine intelligence}
Several observations were made during the development of the MI aspect of the solution.
Initially, the idea for the project was to recognize objects in real-time using machine learning, however as described in \autoref{des:sec:performance}, the object recognition algorithms available, were not performant enough given the hardware limitations.
Instead, the variable motor power problem, described in \autoref{subsec:variablemotorpower}, turned out to be an excellent way to utilize MI for the solution.
The calibration module was implemented using a neural network, which was executed on the NXT.
This proved to be an excellent solution to the variable motor power problem, and a great way to use MI in the solution.
The solution was primarily limited by the hardware used.
A more powerful CPU would make the image-recognition feasible, and this would change the outcome of the system.
In a real world situation, object classification would be important, as actual targets are more difficult to track than red balloons, which have pretty simple attributes.
However, in a real world environment, the hardware would be be more powerful, and in turn making object classification possible.
An important lesson learned from this project was that the usage of MI required analysis of the domain in which the MI is applied to.


\subsection{Real-time requirements}
For the real-time part of the system, several observations were made during the development of the project.
First of all, the amount of tasks that were running on the NXT in order to make it functional were far from enough to give a good CPU utilization using the cyclic executive scheduling model.
In order to optimize the utilization time, parts of the machine intelligence calculations responsible for calculating the amount of power required to move were moved to the NXT for execution.
This resulted in the cyclic executive model being unfeasible, as there was a task that took more than 1 millisecond to run while another task had a deadline of 1 millisecond, thus making it unschedulable. 
To fix this, the scheduling model was changed to fixed priority scheduling with preemption.
The optimal choice for a system like this would be the earliest deadline first scheduling model, but this is not implementable in the nxtOSEK operating system, as this is a fixed priority operating system.

In a real world implementation of \texttt{F.L.A.T.} there are two alternative approaches that makes more sense in terms of CPU utilization: \\
The most obvious solution would be to use a device with a less powerful CPU to avoid that the CPU is idle, or alternatively use a much more powerful device, allowing the whole machine intelligence part of the system to be run on the same device, instead of offloading it to another machine.

In addition to the problems with the CPU utilization, a real world implementation should make use of better components to allow for better precision.

\subsection{Summary}
The primary problems that were experienced during the project period were attributed to the LEGO NXT platform and the nxtOSEK operating system.
Likewise, the initial thoughts for the machine intelligence were deemed infeasible due to the processing power required to perform object recognition and classification. 
In conclusion, the project should be possible to apply to a real world situation given the right hardware.
